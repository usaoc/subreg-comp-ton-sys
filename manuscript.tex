% -*- TeX-engine: luatex; -*-
% Copyright (C) 2023 Wing Hei Chan

% This work is licensed under a Creative Commons
% Attribution-ShareAlike 4.0 International license.  To view a copy of
% this license, visit
% <https://creativecommons.org/licenses/by-sa/4.0/>.

\documentclass[12pt, a4paper]{report}
\usepackage{imakeidx}
\usepackage{setspace}
\usepackage[type=CC, modifier=by-sa, version=4.0]{doclicense}
\usepackage[margin=1in]{geometry}
\usepackage{lua-widow-control}
\usepackage{microtype}
\usepackage{mathtools}
\usepackage{unicode-math}
\usepackage[style]{abstract}
\usepackage[style=unified]{biblatex}
\usepackage[style=british]{csquotes}
\usepackage{datetime}
\usepackage{enumitem}
\usepackage{titlesec}
\usepackage{ebproof}
\usepackage{tikz}
\usepackage[linguistics]{forest}
\usepackage{linguex}

\onehalfspacing
\setlength{\jot}{0pt}
\setlist{noitemsep}
\titleformat{\chapter}{\normalfont\huge\bfseries}{\thechapter}{1em}{}
\titlespacing*{\chapter}{
  0pt}{3.5ex plus 1ex minus .2ex}{2.3ex plus .2ex}

\setmainfont{Libertinus Serif}
\setmathfont{Libertinus Math}
\setmonofont[Scale=MatchUppercase]{Iosevka}

\addbibresource{./reference.bib}

\makeindex[intoc]

\newdateformat{monthyyyy}{\monthname[\THEMONTH] \THEYEAR}

\DeclareNameWrapperFormat{labelname:poss}{#1's}
\newrobustcmd*{\posscitealias}{%
  \AtNextCite{\DeclareNameWrapperAlias{labelname}{labelname:poss}}}
\newrobustcmd*{\posscite}{\posscitealias\textcite}

\usetikzlibrary{automata}
\tikzset{auto}
\forestset{auto/.style={for root=baseline, for tree={calign=first}}}

\newcommand{\context}{\mathrel{/}}
\newcommand{\cuhk}{The Chinese University of Hong Kong}
\newcommand{\gap}{\underline{\hspace{1em}}}
\newcommand{\shift}{\hspace*{1em}}
\newcommand{\textemph}[1]{\textsc{#1}}
\newcommand{\textfeat}[1]{\textsc{#1}}
\newcommand{\textfor}[1]{\textit{#1}}
\newcommand{\textform}[1]{\textit{#1}}
\newcommand{\textgls}[1]{`#1'}
\newcommand{\textphon}[1]{[#1]}
\newcommand{\textterm}[1]{\textsc{#1}\index{#1}}

\renewcommand{\implies}{\Rightarrow}
\renewcommand{\ExLBr}{(\thechapter.}
\renewenvironment{flushleft}{\raggedright}{}

\begin{document}
\pagenumbering{alph}
\begin{titlepage}
  \vspace*{\fill}
  \begin{center}
    \begin{Large}
      \bfseries
      Subregular Complexity of Tonal Systems:

      Case Studies of Chinese Languages
    \end{Large}

    by

    CHAN Wing Hei

    under

    Professor LAI Yee King Regine

    \bigskip

    A Thesis Submitted in Partial Fulfilment of

    the Requirement for the Degree of

    Bachelor of Arts

    in

    Linguistics

    \bigskip

    \cuhk

    \monthyyyy\formatdate{08}{05}{2023}
  \end{center}
  \pretocmd{\doclicenseLongText}{%
    Copyright \copyright\ 2023 Wing Hei Chan\par}{}{}
  \doclicenseThis
  \vspace*{\fill}
\end{titlepage}

\cleardoublepage
\pagenumbering{roman}

\chapter*{Abstract}
\addcontentsline{toc}{chapter}{Abstract}
Phonological transformations based on finite-state transducers have
received an important status since their introduction under the guise
of rewrite rules.  Finite-state transducers represent regular
relations, which are closed under composition.  This shows that an
apparently complex system of phonological rules can be compiled into a
single finite-state transducer.  On the other hand, regular relations
are still too powerful as a probable upper bound of the complexity of
phonological transformations.  For example, regular relations are able
to encode such a phonological transformation that only applies to an
even number of terms, obviously unattested as an adequate rule.

Recently, studies of the so-called subregular classes, that is,
classes even weaker than regular relations, have proved to be a more
probable upper bound.  Such classes are conceptualized as a hierarchy,
not unlike the traditional Chomsky hierarchy in formal language
theory.  The remaining problem is how many phonological phenomena are
explainable and, more importantly, \textemph{not} explainable by each
class.  In this inquiry, tonal systems post an interesting challenge,
and thus require a more in-depth investigation.

Traditionally, tonal systems are accounted for by nonlinear phonology,
which involves the use of multiple phonological tiers and their
associations.  This process can be carefully separated into two ideas:
%
\begin{enumerate}
\item Tones can be independently represented on a separate tier just
  as segments do;
\item The associations between phonological tiers can be manipulated
  in a nontrivial way.
\end{enumerate}
%
In this thesis, the two ideas are examined under the lens of attested
tonal systems found in Chinese languages with vastly different tonal
transformations.  Their implication applies not only to tonal systems,
but also to the interfaces between linguistic levels in general.

\tableofcontents

\cleardoublepage
\pagenumbering{arabic}

\chapter{Theoretical Background}
This chapter introduces the theoretical foundations needed for the
interpretation of tonal systems.  The theories are presented from two
perspectives, namely theoretical and computational phonology, which
utilize different methodologies.  Although this thesis uses a
computational approach, the theoretical implication requires an
understanding of relevant approaches as well.

Specifically, the focus is placed on phonological
\textterm{transformations} as opposed to \textterm{phonotactics}.
Their distinction corresponds to the distinction between transducers
and acceptors in automata theory, which reflect translation and
membership problems respectively.  Loosely speaking, a transducer is a
function of type \(\alpha \to \beta \to \mathit{Boolean}\), while an
acceptor of type \(\alpha \to \mathit{Boolean}\), that is, a
predicate.\footnote{For deterministic transducers, the types are
  isomorphic to \(\alpha \to \beta\).}\footnote{Unless a gradient
  interpretation of phonology is required, in which case the output
  type can be, for example, an interval type.}  The latter will be
briefly mentioned as the theories are introduced.

Given the theoretical background, this chapter then explains the
rationales behind the chosen methodology as well as the expected
results.  In particular, this thesis has the goal of clarifying the
unique operations on autosegmental representations that empower them
to account for certain tonal transformations.

\section{Theoretical Phonology}
This section details the presentations of phonological transformations
in theoretical phonology.  Phonological transformations have been
employed by two allegedly divergent formalisms, often named linear and
nonlinear phonology.  As explained in the later section on the
computational perspective, this categorization fails to recognize the
actual properties that render computationally more complex formalisms
necessary.  Nonetheless, this section will follow the traditional
categorization.

This section also discusses the organization of grammar.  As will
become clear once computational models are introduced, organization
only matters in the compositions of phonological transformations,
contrary to the common conception that cyclicity and ordering are
inherent properties of phonological systems.  Lexical phonology is
mentioned for its significance in the arguments against cyclicity.

The discussion of \textterm{optimality theory} as in
\textcite{ps93otcigg} is intentionally omitted.  The reason is not
that optimality theory is a worse formalism, but merely that its
formalization results in computational devices out of the scope of
this thesis.  Specifically, optimality theory, due to its
constraint-optimizing nature, requires constraint optimizers rather
than transducers, which are potentially powerful enough to
overgenerate.  For such an implementation based on dynamic
programming, refer to \textcite{t95cot}.  For discussions of the
computational complexity of optimality theory, refer to
\textcite{e97egpot, i06sptotci, hkr09ecot}, among others.

\subsection{Linear Phonology}
The earliest discussion of phonological transformations dates back to
\textcite{ch68spe}, where rules are in the form of context-sensitive
rewrite rules.  For example, a vowel nasalization rule can be in the
form

\ex. \(\text{V} \to [+\text{nasal}]
\context \text{\gap}[+\text{nasal}]\)

read as \enquote{nasalize a vowel immediately before a nasal sound}.
Under the usual notation of rewrite rules, this is written as

\ex. \(\text{V}[+\text{nasal}] \to
(\text{V} \cup [+\text{nasal}])[+\text{nasal}]\)

understood as \enquote{rewrite a vowel immediately before a nasal
  sound to the union of the vowel and the singleton set of the nasal
  feature}.  It should be noted that this apparently context-sensitive
form is by no means actually context-sensitive.  What it means is
that, given the Chomsky hierarchy \parencite{c59cfpg}

\ex. \(\text{Recursively enumerable} \supset
\text{Context-sensitive} \supset
\text{Context-free} \supset
\text{Regular}\)

it is not true that phonological transformations form a superset of
context-free relations.  This is shown in the later section on regular
relations.

Attention should be paid to \enquote{linear}.  One interpretation is
that phonological transformations apply to a linear sequence of
segments.  This assumes that segmental phonology, the sort that is
dealt with above, is disjoint from nonsegmental phonology.  This
assumption is computationally unsound, as nonsegmental phonology does
not necessarily require a \enquote{nonlinear} account, and conversely
segmental phonology may require one.  Another interpretation has to do
with the notion of locality, which is an often misunderstood notion in
theoretical phonology.

A relevant concern is the representations of tonal features.  Indeed,
due to the \enquote{linear} nature of the representations, tonal
features have nowhere else to fit in but the same feature sets as the
segmental features.  Consider, for example, \posscite{w67pft}
interesting characterization of Xiamen tone cycle:

\ex. \([\mathop{\alpha}\text{high}, \mathop{\beta}\text{fall}]
\to [\mathop{\beta}\text{high}, \mathop{-\alpha}\text{fall}]\)

This rule not only uses the same \textemph{form} as segmental rules,
but also the same \textemph{representation}.  Of course, it is not a
problem if the rule explains the tone cycle well.\footnote{See,
  however, \posscite{c00tspcd} criticism of the characterization.}  On
the other hand, a nonlinear account allows more freedom both in the
form and representation of the rule.  Readers are thus reminded that
representation is a separate matter from the computational complexity
of the formalism \textemph{independent} of representation.

\subsection{Nonlinear Phonology}
Despite the remarks made above, nonsegmental phonology is precisely
what has inspired the development of nonlinear phonology, also known
as autosegmental phonology.  Early applications of nonlinear phonology
can be found in \textcite{c76vhngpam, g76ap}, among others.  Under
nonlinear phonology, phonological representations consist of multiple
associated tiers, each tier with its own linear sequence of terms.
For example, a syllable \textphon{ma} with a high level tone can be
represented as

\ex.
\begin{forest}
  auto, [ma [\(\sigma\) [H] [H]]]
\end{forest}

where \(\sigma\) stands for a syllable and H stands for a high tone.
Consequently, phonological transformations are generalized to operate
on associations.  An unbounded tone spreading rule can be given as:

\ex. \(
\begin{forest}
  auto, [T [\(\sigma\)] [\(\sigma\), no edge] [\(\sigma\), no edge]]
\end{forest}
\to
\begin{forest}
  auto, [T [\(\sigma\)] [\(\sigma\)] [\(\sigma\), no edge]]
\end{forest}
\to
\begin{forest}
  auto, [T [\(\sigma\)] [\(\sigma\)] [\(\sigma\)]]
\end{forest}\)

An implicit assumption is that the rule applies iteratively from left
to right.  This intuition is, surprisingly, captured equally well by
linear transformations.  An equally acceptable formulation is the
linear transformation

\ex. \(\text{T}\varepsilon\varepsilon
\to \text{TT}\varepsilon
\to \text{TTT}\)

where \(\varepsilon\) stands for the empty term, assuming the
transformation also applies iteratively from left to right.

The pair of examples above suggests that the power of nonlinear
phonology comes not from the representations, nor from the capability
of iterative application.  Instead, it comes from the ability to
encode the associations in the transformations, as explicified by the
computational models.  Iterative application, on the other hand,
concerns the computational nature of the rules, which is reflected by
the mappings their corresponding process produces.

A point should be made that iterative application has little to do
with the organization of grammar.  An iteratively applied phonological
transformation is neither inherently associated with cyclicity nor
with ordering.  In fact, it is not necessarily iterative at all under
an alternative representation.  The only thing relevant is the
\textemph{process} the rule generates as embodied by the equivalent
computational device.

\subsection{Organization of Grammar}
What organization of grammar concerns is, then, \textemph{not} the
computational nature of phonological transformations, but rather the
use of linguistic formalisms to express computationally equivalent
models.\footnote{This is not to deny that linguistic formalisms differ
  in their expressivity in a nontrivial manner.  Expressivity and
  computability are orthogonal, as the famous \enquote{Turing tarpit}
  analogy puts it \parencite{p82ep}.}  This is reminiscent of the use
of programming paradigms to express computationally equivalent
programs given their Turing completeness, except we expect a much
weaker upper bound in phonological models.

The concept of \textterm{cyclicity}, that is, repeatedly applying a
phonological transformation until it is no longer applicable, played
an important role in early formulations of nonsegmental phonology.
The idea is that a series of rules is linearly sequenced as
\(R_{1}, \ldots, R_{n}\), where \(R_{n}\) is a rule that creates new
contexts for the previous rules, often conceived as a \enquote{bracket
  erasure} rule.  This brings two problems:
%
\begin{enumerate}
\item The end result of rule applications is tightly coupled with how
  contexts are represented;
\item It is not clear how to apply different sequences of rules at
  each level of representation.
\end{enumerate}
%
The former is hardly an inherent or unique problem of cyclicity.  The
latter, in comparison, is a more serious problem, as it prohibits an
otherwise modular organization of grammar by the virtue of
\enquote{rules bloat}.  Alternatively, \textterm{lexical phonology}
advocates for a \enquote{stratified} approach \parencite{k82cplp},
thus achieving a more modular organization.  This thesis adopts the
general approach of lexical phonology without assuming any form of the
lexicalist hypothesis.

The \textterm{ordering} of phonological transformations, on the other
hand, is a more universal problem.  Even simply modeling phonological
transformations as functions already leads to the ordering problem, as

\ex. \(\neg(\forall f, g\ldotp f \circ g \implies g \circ f)\)

That is, function composition is noncommutative.  Therefore, this
thesis does not expect this problem to be easily solved.

\section{Computational Phonology}
This section introduces computational formalization of phonological
transformations in terms of regular relations, in particular
subregular functions.  The main method of computationally formalizing
phonological transformations concerns the use of machines in the sense
of automata theory.  In particular, finite-state machines are used for
their limited computational complexity.  They are able to encode
regular relations, but not context-sensitive or even context-free
relations due to their lack of a stack.

Logical encodings of regular relations are also introduced along with
their transductions.  Encoding regular relations in terms of logical
formulae facilitates the understanding of their subparts as compared
to transducers and enables the evaluation of their computational
complexity as first-order, quantifier-free, among others.

\subsection{Regular Relation}
The apparently context-sensitive phonological rules have the important
constraint that all terms are \textterm{terminal}, that is, not able
to be further rewritten into themselves, directly or indirectly.  As
\textcite{c59cfpg} points out, the distinguishing feature of
context-free languages is exactly the ability to self-embed, which
requires recursively rewriting \textterm{nonterminal} terms into
themselves preceded and followed by other terms.  Formally, a grammar
is self-embedding iff

\ex. \(\exists A, \varphi, \psi\ldotp A \to^{+} \varphi A \psi\)

where neither \(\varphi\) nor \(\psi\) is the identity term and
\(\to^{+}\) is the transitive closure of the rewrite relation.
Clearly, phonology does not permit such a structure.
\textcite{kk94rmprs} therefore suggests modeling phonological
transformations as \textterm{regular} relations.

Regular relations are relations whose domains and ranges are regular
languages.  An artificial regular relation, for example, is a relation
that rewrites any sequence that has an \textemph{even} number of \(a\)
to a sequence with each \(a\) replaced by \(b\).  This is an
unattested phonological transformation, but it is a valid regular
relation, showcasing that regular relations are still computationally
too complex for modeling phonological transformations.

Regular relations can be represented as the equivalent
\textterm{finite-state} transducers.  Finite-state transducers are
machines that transit between a finite number of states according to
the currently read term, where transitions emit the output terms.  To
put it another way, \textterm{acceptors} are machines that accept
\textemph{one} sequence of terms, while \textterm{transducers} are
ones that accept \textemph{two} sequences of terms, namely the input
and output sequences.  The artificial regular relation above requires
a \textterm{nondeterministic} finite-state transducer that either
rewrites \(a\) to \(b\) or does not, whose validity is only determined
after the whole input sequence is read:

\ex.
\begin{tikzpicture}[baseline=(bow.base)]
  \node[state] at (0, 0) (bow) {\(q_{1}\)};
  \node[state, accepting] at (2, 1) (noop) {\(q_{2a}\)};
  \node[state] at (4, 1) (invalid) {\(q_{3a}\)};
  \node[state] at (2, -1) (write) {\(q_{2b}\)};
  \node[state, accepting] at (4, -1) (valid) {\(q_{3b}\)};
  \path[->]
  (bow) edge [loop left] node {\(\text{?}:\text{?}\)} ()
  (bow) edge node {\(a:a\)} (noop)
  (noop) edge [loop above] node {\(\text{?}:\text{?}\)} ()
  (noop) edge [bend left] node {\(a:a\)} (invalid)
  (invalid) edge [loop above] node {\(\text{?}:\text{?}\)} ()
  (invalid) edge [bend left] node {\(a:a\)} (noop)
  (bow) edge ['] node {\(a:b\)} (write)
  (write) edge [loop below] node {\(\text{?}:\text{?}\)} ()
  (write) edge [bend left] node {\(a:b\)} (valid)
  (valid) edge [loop below] node {\(\text{?}:\text{?}\)} ()
  (valid) edge [bend left] node {\(a:b\)} (write);
\end{tikzpicture}

This artificial regular relation is unattested, which means the
transducer overgenerates.  Ideally, two additional constraints are
desired:
%
\begin{enumerate}
\item The transducers be \textterm{deterministic} to exclude multiple
  outputs;
\item The transducers be \textterm{local} to limit the required memory
  resource.
\end{enumerate}
%
These two constraints, in turn, lead to subregular functions.

Alternatively, regular relations can be represented as logical
transductions, specifically \textterm{monadic second-order} ones as
those in \textcite{eh01mdsttft}, which are actually able to represent
nonregular relations.\footnote{An example is the first-order definable
  total reduplication \(w \to ww\), whose output language is included
  in the class of tree-adjoining languages \parencite{sw94efecg}.}
Monadic second-order logic allows quantification over sets, which is
needed for the definition of general precedence \(<\) based on
immediate precedence \(\vartriangleleft\):\footnote{This is to assume
  the relations are nonrecursive.}

\ex.
\a. \(\mathit{Transitive}(X) \coloneq \forall x, y\ldotp
(x \in X \land x \vartriangleleft y) \implies y \in X\)
\b. \(x < y \coloneq \forall X\ldotp
(x \in X \land \mathit{Transitive}(X)) \implies y \in X\)

This is understood as \enquote{\(x\) precedes \(y\) iff every
  transitive set with respect to immediate precedence that includes
  \(x\) also includes \(y\)}, which means the transitive closure also
includes \(y\).  A relation is then defined that expresses that an
\(a\) immediately precedes another \(a\) ignoring any non-\(a\)
between them:

\ex. \(x \vartriangleleft_{a} y \coloneq a(x) \land a(y) \land x < y
\land \neg(\exists z\ldotp a(z) \land x < z \land z < y)\)

The notion of the evenness of the number of \(a\) is finally defined:

\ex.
\a. \(\mathit{Odd}(x) \coloneq (a(x) \land (\forall y\ldotp
y < x \implies \neg a(y))) \lor (\exists y\ldotp
\mathit{Even}(y) \land y \vartriangleleft_{a} x))\)
\b. \(\mathit{Even}(x) \coloneq \exists y\ldotp
\mathit{Odd}(y) \land y \vartriangleleft_{a} x\)

Accordingly, the transduction is defined:\footnote{Observant readers
  may notice that an efficient transducer implementing said
  transduction reads the input \textemph{twice}, dispatching on the
  number of \(a\) in the first pass.  This is due to the
  correspondence between monadic second-order logic and two-way
  finite-state transducers.}

\ex. \(b'(x) \coloneq b(x) \lor (a(x) \land (\exists y\ldotp
\mathit{Even}(y) \land (\forall z\ldotp y < z \implies \neg a(z))))\)

Similar to how finite-state transducers are constrained, logical
transductions of the sort above are avoided if
%
\begin{itemize}
\item Quantification over sets is disallowed; or
\item Any quantification is disallowed.
\end{itemize}
%
They result in \textterm{first-order} logic and its
\textterm{quantifier-free} version respectively, which correspond to
various subregular functions.

Despite their inadequate computational complexity, regular relations
are useful as they are closed under composition, meaning that

\ex. \(\forall P, Q\ldotp (\mathit{Regular}(P) \land
\mathit{Regular}(Q)) \implies \mathit{Regular}(P \circ Q)\)

This enables the possibility to compile an apparently complex system
of phonological rules into a single finite-state transducer.

\subsection{Subregular Function}
The most trivial class of \textterm{subregular} functions is without a
doubt the class of finite functions, which is disfavored in this
thesis for the reasons mentioned in \textcite{s93wimpatlai}.  Rather,
by enforcing the aforementioned constraints, several nontrivial
classes of subregular functions can be subdivided.  If the
finite-state transducers are required to be deterministic, the
resulting subregular functions are \textterm{subsequential} functions,
further classified into left and right variants depending on the
direction the input sequence is read \parencite{ch12bcsimr, hl13vhs}.
This class is not particularly interesting for the purposes of this
thesis, as it lacks a meaningful notion of locality.

The concept of \textterm{strict locality} is defined for regular
languages, which concerns the recognition of substrings
\parencite{rp11apresh, rhfhlw13csc}.  Borrowing this concept, input
and output strictly local functions can be defined over
\textterm{linear} representations, which \enquote{remember} a finite
length of input and output substrings respectively
\parencite{ceh14lslsf}.  Similar to subsequential functions, output
strictly local functions can be further classified into left and right
variants \parencite{ceh15oslf}.

An example of input strictly local function is the function that
rewrites \(a\) to \(b\) whenever the last input is \(a\).  That is,

\ex. \(a^{n+1} \to ab^{n}\)

where \(n \in \mathbb{N}\), with the equivalent transducer

\ex.
\begin{tikzpicture}[baseline=(loop.base), initial text=\(a:a\)]
  \node[state, initial] at (0, 0) (loop) {\(a\)};
  \path[->] (loop) edge [loop right] node {\(a:b\)} ();
\end{tikzpicture}

where the remembered input term is encoded in the state.  More
conveniently, it can also be represented as the logical transduction

\ex.
\a. \(a'(x) \coloneq a(x) \land \neg(a(\mathit{pred}(x)))\)
\b. \(b'(x) \coloneq a(x) \land a(\mathit{pred}(x))\)

where \(\mathit{pred}\) is the predecessor function.

Similarly, an example of left output strictly local function is the
function that rewrites \(a\) to \(b\) whenever the last output is
\(a\).  That is,

\ex.
\a. \(a^{2n} \to (ab)^{n}\)
\b. \(a^{2n+1} \to (ab)^{n}a\)

with the equivalent transducer

\ex.
\begin{tikzpicture}[baseline=(a.base), initial text=\(a:a\)]
  \node[state, initial] at (0, 0) (a) {\(a\)};
  \node[state] at (2, 0) (b) {\(b\)};
  \path[->]
  (a) edge [bend left] node {\(a:b\)} (b)
  (b) edge [bend left] node {\(a:a\)} (a);
\end{tikzpicture}

where the remembered output terms are encoded in the states.  When it
is represented as a logical transduction, however, there occurs
\textterm{recursive} logical formulae,\footnote{Readers familiar with
  fixed-point theorems may expect to see a fixed-point logic
  undermining such recursive logical formulae.  The technical details
  can be found in \textcite{koj18taolns, cj19qlfpfp}.} resulting in a
recursive strictly local function:

\ex.
\a. \(a'(x) \coloneq a(x) \land \neg(a'(\mathit{pred}(x)))\)
\b. \(b'(x) \coloneq a(x) \land a'(\mathit{pred}(x))\)

As \textcite{cj21iolr} conjectures, recursive logical transductions
comprise precisely the logical characterization of output strictly
local functions.

Logical encodings enable the generalization of strict locality over
\textterm{autosegmental} representations \parencite{cj19aislf}.  Under
a predicate logic, associations are understood as a binary relation
\(\mathit{Assoc}\) where \(\mathit{Assoc}(a, b)\) indicates that \(a\)
and \(b\) are associated, where \(a\) and \(b\) are on separate
tiers.\footnote{A question can be asked whether \(\mathit{Assoc}\) is
  symmetric.  Although this question has no particular bearing,
  \(\mathit{Assoc}\) is assumed to be antisymmetric to allow the
  interpretation of autosegmental representations as directed graphs.}
For example, the unbounded tone spreading rule can be defined as the
recursive logical transduction

\ex. \(\mathit{Assoc}'(x, y) \coloneq \mathit{Assoc}(x, y) \lor
\mathit{Assoc}'(x, \mathit{pred}(y))\)

Is it then easy to see why it works equally well for linear
representations.  \(\mathit{Assoc}\) is merely used to represent the
presence of tones:

\ex.
\a. \(\text{T}(x) = \exists y\ldotp \mathit{Assoc}(y, x)\)
\b. \(\text{T}'(x) = \exists y\ldotp \mathit{Assoc}'(y, x)\)

This makes it possible to surject the autosegmental transformation
into a linear one:

\ex.
\a. \(\text{T}'(x) \coloneq
\text{T}(x) \lor \text{T}'(\mathit{pred}(x))\)
\b.
\begin{prooftree}
  \hypo{\mathit{Assoc}'(a, x) =
    \mathit{Assoc}(a, x) \lor \mathit{Assoc}'(a, \mathit{pred}(x))}
  \infer1{(\exists y\ldotp \mathit{Assoc}'(y, x)) =
    (\exists y\ldotp \mathit{Assoc}(y, x) \lor
    \mathit{Assoc}'(y, \mathit{pred}(x)))}
  \infer1{(\exists y\ldotp \mathit{Assoc}'(y, x)) =
    ((\exists y\ldotp \mathit{Assoc}(y, x)) \lor
    (\exists y\ldotp \mathit{Assoc}'(y, \mathit{pred}(x))))}
  \infer1{\text{T}'(x) = \text{T}(x) \lor \text{T}'(\mathit{pred}(x))}
\end{prooftree}

In other words, the linear transformation is \textterm{embedded}.  The
existential introduction reflects the fact that the transformation
only cares about the \textemph{existence} of any term on the tonal
tier that is associated with the input term.


Strictly local logical tranductions, as shown thus far, satisfy both
first-orderness and quantifier-freeness.  This is not a coincidence,
but rather the expected outcome.  By disallowing quantification,
logical formulae are allowed to refer to terms other than the input
terms only by the means of the predecessor and successor functions,
which in turn allude to the notion of locality.

\section{Preliminary Goal}
Chinese languages have been famous for their rich repertoire of tonal
systems and relevant transformations.  In particular, tonal
transformations in Chinese languages often interact with the
morphosyntactic representations.  This fact has inspired analyses that
require morphosyntactic encodings in phonological representations.
This thesis expects to formulate such techniques in terms of strictly
local functions.

Another question this thesis aims to answer is why autosegmental
representations appear necessary.  Indeed, as later shown,
autosegmental representations impose the requirement that phonological
representations be graphs as opposed to linear sequences of terms.
This question is impossible to answer without assessing what
autosegmental representations are useful for.

Finally, this thesis attempts to supply a theoretical interpretation
of the computational analyses.  The theoretical implication of
strictly local functions, this thesis believes, is more general than
the study of computational complexity.  Potentially, it relates to the
more difficult problem of the interfaces between linguistic levels.  A
complete answer to this question requires a better understanding of
the primitives and operations each level allows.

\chapter{Boundary of Transformation}
With the computational preliminaries in place, we are now ready to
analyze the attested tonal transformations found in Chinese languages.
This chapter specifically discusses transformations involving
boundaries that can block the applications.  Despite the simple facets
of transformations \textfor{per se}, the existence of boundaries
raises certain important questions, such as where, when, and how they
come into play.  This is especially problematic under linear models,
as hierarchical interpretations of boundaries often require more than
linear sequences.

\section{Representation of Boundary}
The linear understanding of \textterm{boundaries} is simply certain
terms that never surface, yet block any transformation that may
otherwise apply.  A boundary term is always output \textemph{as is},
therefore failing to satisfy any condition when it precedes or
succeeds the input term in question.  Consider again the unbounded
tone spreading rule:

\ex. \(\acute{\sigma}'(x) \coloneq \acute{\sigma}(x) \lor
(\sigma(x) \land \acute{\sigma}'(\mathit{pred}(x)))\)

A boundary term certainly cannot satisfy \(\sigma(x)\), and thus
\(\acute{\sigma}'(\mathit{pred}(x))\) cannot be true when it precedes
the input term.  This quickly fails to account for more complicated
rule interactions, where boundaries may \textemph{change} to create
new contexts.  Recall that cyclicity assumes that each cycle creates
new contexts until fixed point---this intuition leads to analyses that
delete and insert new boundaries.

An alternative view of boundaries is \textterm{groupings}, which can
render the transformation nonregular.  Even if we ignore this nature
of arbitrary groupings by restricting self-embedding in some manner,
there is still the issue of determining whether two terms are in the
same group.  Suppose groupings are delimited by conventional brackets,
to determine whether two terms are in the same group, the transducer
potentially needs to traverse the whole input sequence.  This is more
apparent in the case of autosegmental representations, where groupings
are represented as trees, which are a subset of directed
graphs.\footnote{Such trees with a limited depth are known as
  \enquote{prosodic hierarchy}, for which see \textcite{nv86pp}.} The
relation that decides whether two terms are in the same group is then

\ex. \(\mathit{SameGroup}(x, y) \coloneq \exists z\ldotp
\mathit{Assoc}(z, x) \land \mathit{Assoc}(z, y)\)

assuming that groupings do not nest.  As usual, nestable groupings
require the transitive closure of \(\mathit{Assoc}\), which is monadic
second-order definable.

If we entirely discard the idea of explicitly encoding boundaries in
input sequences, we need to derive the same processes as before,
thereby implicitly encoding boundaries in some sense.  What boundary
terms achieve is essentially applying a single transformation to
multiple linear sequences and combining the outputs through some
associative operation, say concatenation.  In other words, the
transformation is \enquote{lifted} to operate on multiple linear
sequences.  By doing so, we modularize the transformation into a
simpler transformation and a universal lift operation.\footnote{This
  operation is precisely a functor, or loosely speaking function of
  type
  \((\alpha \to \beta) \to \mathbb{F}\alpha \to \mathbb{F}\beta\),
  commonly known as \(\mathit{map}\).  \(\mathbb{F}\) should produce a
  type whose values can be \enquote{folded} in the sense of
  \textcite{h99tuem} under a monoid.}

However, groupings are much different, as they create input
\textemph{trees}.  Input trees, by their inductive nature, can contain
more trees, which are in turn input trees to a potentially new
transducer with appropriate hierarchical notions encoded.  It not only
generates a recursive process, but also creates a leaky abstraction as
the transducer must be aware of the whole input up to a certain level.
This defeats the purpose of having multiple linguistic levels in the
first place.

The ideal scenario is that each linguistic level drives phonological
realization by feeding the next level inputs and combining the
outputs.  This eliminates the need of boundaries altogether, while
implicitly preserving their essence.  This is also much what cyclicity
really aims to achieve, namely that each cycle further transforms the
input sequence with one layer of groupings removed.  What this means
in practice will be demonstrated in the following sections as relevant
linguistic data are covered.

\section{Standard Chinese: Tone 3 Sandhi}
Tone 3 sandhi is a tonal transformation found in Standard Chinese,
where tone 3 is dissimilated to tone 2 before another tone 3.  This
process can be described as simultaneously applied, as seen in the
following data:

\ex.
\a. \(\text{\textform{zhǎnlǎn}} \to
\text{\textform{zhánlǎn} \textgls{exhibit}}\)
\b. \(\text{\textform{zhǎnlǎnguǎn}} \to
\text{\textform{zhánlánguǎn} \textgls{exhibition hall}}\)
\b. \(\text{\textform{zhǎnlǎnguǎn lǐ}} \to
\text{\textform{zhánlánguán lǐ} \textgls{in exhibition hall}}\)

Therefore, a logical transduction

\ex. \(\acute{\sigma}'(x) \coloneq \acute{\sigma}(x) \lor
(\check{\sigma}(x) \land \check{\sigma}(\mathit{succ}(x)))\)

resulting in the mapping

\ex. \(\check{\sigma}^{n+1} \to \acute{\sigma}^{n}\check{\sigma}\)

should be enough to describe the process.  It should be emphasized
that this process is linear in its computational nature,
\textemph{regardless} of the specific representation used here for
convenience.  Indeed, even if the tones are represented on a separate
tier with multiple terms, the transformation is still linear and,
importantly, strictly local, albeit with a larger memory
size.\footnote{To decide the memory size, it suffices to calculate the
  number \(n + m\), where \(n\) and \(m\) are respectively the highest
  degrees of composition of the predecessor and successor functions in
  the normalized transduction.}  Namely, the mapping will then be

\ex. \((\text{LL})^{n+1} \to (\text{LR})^{n}\text{LL}\)

ignoring the optional alternative realization of tone 3 when it is
lengthened \parencite{d99mstems}.  The fundamental reason is that the
associations are not affected by the transformation, that is,
\(\mathit{Assoc}\) is never mentioned in the transduction.

The transformation becomes more interesting when boundaries are
concerned.  The data

\ex.
\a. \(\text{\textform{gǒu bǐ mǎ xiǎo}} \to
\text{\textform{góu bǐ má xiǎo}}\)
\bg. gǒu bǐ mǎ xiǎo\\
dog than horse small\\
\trans\textgls{Dogs are smaller than horses.}

show that certain tone 3 does \textemph{not} change to tone 2.  This
seems to suggest that an alternative transduction

\ex. \(\acute{\sigma}'(x) \coloneq \acute{\sigma}(x) \lor
(\check{\sigma}(x) \land \check{\sigma}'(\mathit{succ}(x)))\)

resulting in the mapping

\ex.
\a. \(\check{\sigma}^{2n} \to (\acute{\sigma}\check{\sigma})^{n}\)
\b. \(\check{\sigma}^{2n+1} \to
\check{\sigma} (\acute{\sigma}\check{\sigma})^{n}\)

is in effect.  This is not true, as the data

\ex.
\a. \(\text{\textform{xiǎogǒu bǐ xiǎomǎ xiǎo}} \to
\text{\textform{xiáogóu bǐ xiáomá xiǎo}}\)
\bg. xiǎo-gǒu bǐ xiǎo-mǎ xiǎo\\
small-dog than small-horse small\\
\trans\textgls{Puppies are smaller than ponies.}

clearly do not adhere to the mapping.  Moreover, the data

\ex.
\a. \(\text{\textform{xiǎo zhǐlǎohǔ}} \to
\{\text{\textform{xiǎo zhíláohǔ}},
\text{\textform{xiáo zhǐláohǔ}},
\text{\textform{xiáo zhíláohǔ}}\}\)
\bg. xiǎo zhǐ-lǎohǔ\\
small paper-tiger\\
\trans\textgls{small paper tiger}

show that certain tone 3 \textemph{optionally} changes to tone 2.  To
account for such contrasts, various analyses have been developed.  Two
such analyses will be discussed, where one is representative for
\enquote{syntactic} analyses and the other for \enquote{prosodic}
analyses.

\subsection{Syntactic Analysis}
\textcite{c00tspcd} argues for a analysis under which feet are built
based on syntactic structures, which in turn govern the application of
tone 3 sandhi.  As \textcite{d07psc} states, Although this analysis
includes a notion of feet, they are nonetheless not related to prosody
given their insensitivity to stress.  More crucially, the tone 3
sandhi rule makes reference to the syntactic structure for the
applicability.  Therefore, it is considered a \enquote{syntactic}
analysis here.

The basic idea is that syntactic branching leads to different foot
structures.  For example, a left branching syntactic structure leads
to a left foot:

\ex.
\a. \(\text{(\textform{mǎi hǎo}) \textform{jiǔ}} \to
\text{(\textform{mái háo}) \textform{jiǔ}}\)
\bg. [[mǎi hǎo] jiǔ]\\
\phantom{[[}buy good wine\\
\trans\textgls{to finish buying wine}

Tone 3 sandhi applies to \textform{mǎi} as it is followed by another
tone 3 syllable in the same foot.  The curious part is \textform{hǎo},
which is followed by another tone 3 syllable in a \textemph{higher}
group.  A right branching syntactic structure, on the other hand,
leads to a right foot:

\ex.
\a. \(\text{\textform{mǎi} (\textform{hǎo jiǔ})} \to
\{\text{\textform{mǎi} (\textform{háo jiǔ})},
\text{\textform{mái} (\textform{háo jiǔ})}\}\)
\bg. [mǎi [hǎo jiǔ]]\\
\phantom{[}buy \phantom{[}good wine\\
\trans\textgls{to buy good wine}

Tone 3 sandhi now only optionally applies to \textform{mǎi}, as it is
followed by another tone 3 syllable in a \textemph{lower} group.
Apparently, this implies that nestable groupings are required.  As
aforementioned, nestable groupings lead to monadic second-order logic
over \textemph{trees}, which is an undesirable property.

Alternatively, we may consider that the trees be transformed to linear
sequences suitable as inputs to the later level.  This can be a simple
recursive process that flattens the input trees using an associative
operation that prefixes boundaries.\footnote{This process can also be
  represented as a strictly local tree transducer, as linear sequences
  are isomorphic to single-pathed graphs, that is, unary-branching
  trees.  See \textcite{ioj20qtt, jh20isltt} for detailed discussions
  of such transducers.}  This tree transformation can be
nondeterministic to account for the optionality, but the level it
feeds, tone 3 sandhi, remains deterministic.\footnote{Under
  nondeterminism, an output sequence that contains a
  \(\check{\sigma}\check{\sigma}\) substring simply fails and
  backtracks.  Although \textcite{c00tspcd} reports several data that
  contain such substrings, they are more likely due to emphatic
  stresses.}  Considering only the nontrivial cases, we have

\ex.
\a. \(
\begin{forest}
  nice empty nodes,
  [[\(v\), baseline] [[\(w\)] [\(z\)]]]
\end{forest}
\to
\#v\#wz\)
\b. \(
\begin{forest}
  nice empty nodes,
  [[[\(v\)] [\(w\)]] [\(z\), baseline]]
\end{forest}
\to
\#\#vwz\)


In some sense, this is as if only opening brackets were preserved.
Syntactic categories are insignificant, as the process only makes
reference to the hierarchical structure.  This, of course, still
requires modification to tone 3 sandhi, as now the input sequences
contain boundaries.  To completely suppress boundaries in the inputs,
we can further split linear sequences by boundaries, resulting in
multiple linear sequences.\footnote{Or, obviously, transform to
  multiple linear sequences in a single pass.  The presentation is
  merely for easier understanding under a graph-transduction
  perspective.}  Tone 3 sandhi is then lifted to operate on these
linear sequences, whose outputs are combined by concatenation.  To put
it simply, this means that

\ex.
\a. \(\#v\#wz \to \tau(v) \cdot \tau(w \cdot z)\)
\b. \(\#\#vwz \to \tau(v \cdot w \cdot z)\)

omitting empty sequences.  Interestingly, the contrast now comes from
the fact that tone 3 sandhi does not preserve concatenation:

\ex. \(\neg(\forall x, y\ldotp
\tau(x \cdot y) = \tau(x) \cdot \tau(y))\)

If this \textemph{did} hold, we would be able to derive the
equivalence of the two mappings due to associativity.

A note should be made regarding cyclicity.  In \posscite{c00tspcd}
original analysis, cyclicity is used to explain both simultaneous and
optional applications, on the assumption that groupings can be
expanded at each level.  However, we have seen how cyclicity is
\textemph{not} an essential property of the analysis by virtue of
modularization, where a tree transformation feeds tone 3 sandhi.

\subsection{Prosodic Analysis}
\textcite{d07psc} proposes an alternative analysis in light of several
perceived inadequacies of the previous syntactic analysis.  The most
significant one is that supposedly same syntactic structures can lead
to different groupings, as

\ex. \(\text{(\textform{něizhǒng}) (\textform{jiǔ hǎo})} \to
\text{(\textform{néizhǒng}) (\textform{jiú hǎo})}\)

must be derived from

\exg. [[[něi-zhǒng] jiǔ] hǎo]\\
\phantom{[[[}which-kind alcoholic.beverage good\\
\trans\textgls{Which kind of alcoholic beverage is good?}

supposing the syntactic structure is correct at all.  This obviously
hinges on specific syntactic analysis, as one can imagine

\exg. [[[něi-zhǒng] [\gap{} jiǔ]] hǎo]\\
\phantom{[[[}which-kind \phantom{[}\textfeat{gap} alcoholic.beverage
good\\
\trans\textgls{Which kind of alcoholic beverage is good?}

being a totally valid alternative analysis that \textemph{does} derive
the correct grouping.\footnote{Justification for a similar analysis
  based on type shifting can be found in \textcite{bs14cnl}, which
  generalizes to all determiners.}  With the addition of gap terms,
the tree transformation requires more elaboration, but this does not
change the computational nature of the linear transformation.
Similarly, some other provided data can arguably be attributed to
alternative syntactic analyses.

Another perceived inadequacy is the insensitivity to stresses, which
is the reason behind the name \enquote{stress-insensitive foot}.  For
example, an emphatic stress as in

\ex. \(\text{\textform{xiǎng} (\textform{MǍI gǔpiào})} \to
\text{\textform{xiǎng} (\textform{MÁI gǔpiào})}\)

seems to favor (force?) a boundary before the emphasized syllable.
Again, this only requires modification to the tree transformation.
Nonetheless, \textcite{d07psc} opts for a more aggressive modification
to both the tree transformation and tone 3 sandhi.

Under the proposal, syntactic trees are assigned stresses according to
the \enquote{information-stress principle}, which covers syntactic
structures assuming that nonheads carry more information.  The
syllables are then grouped into feet based on stresses, where a
stressed syllable must starts a group.  As these feet are considered
metrical due to the relevance of stresses, the analysis is a prosodic
one.

The complication is that tone 3 sandhi must be aware of
\textemph{both} the syntactic structure and prosodic groupings due to
the way \enquote{adjacency} is defined.  Two syllables are said to be
\enquote{adjacent} when they are contained by the same syntactic
constituent and not by separate full feet, where a full foot is a
grouping with more than one syllable.  This notion must be defined
over autosegmental representations:

\ex.
\a. \(\mathit{Adjacent}(x, y) \coloneq \mathit{SameConst}(x, y) \land
\neg\mathit{SepFullFeet}(x, y)\)
\b. \(\mathit{SameConst}(x, y) \coloneq \exists z\ldotp
\mathit{Dom}(z, x) \land \mathit{Dom}(z, y)\)
\b. \(\mathit{SepFullFeet}(x, y) \coloneq \exists z, w\ldotp
z \neq w \land \mathit{FullFoot}(z, x) \land \mathit{FullFoot}(w, y)\)
\b. \(\mathit{FullFoot}(x, y) \coloneq \mathit{Foot}(x, y) \land
(\exists z\ldotp \mathit{Foot}(x, z) \land
(\mathit{pred}(y) = z \lor \mathit{succ}(y) = z))\)

\(\mathit{Dom}\) is the transitive closure of the syntactic
association relation, and \(\mathit{Foot}\) is the prosodic
association relation.  This is overcomplicated as a phonological
process, at least from our point of view.

Moreover, tone 3 sandhi is now said to be a nondeterministic process,
which optionally changes tone 3 before nonadjacent or sandhi tone 3.
In a logical world, this can be achieved by an operator that
introduces choice points, say \posscite{m61bmtcpr} \(\mathit{amb}\)
operator:

\ex.
\a. \(\acute{\sigma}'(x) \coloneq \acute{\sigma}(x) \lor
(\check{\sigma}(x) \land \check{\sigma}(\mathit{succ}(x))
\land \neg\mathit{MaybeSkip}(x))\)
\b. \(\mathit{MaybeSkip}(x) \coloneq
\mathit{amb}(\neg\mathit{Adjacent}(x, \mathit{succ}(x)),
\check{\sigma}(\mathit{succ}(\mathit{succ}(x))), \bot)\)

We see how convoluted the analysis becomes once the phonological
process is no longer subregular.  Even though tone 3 sandhi is one of
the simplest phonological processes, the \textemph{condition} it
applies heavily depends on syntactic information.  Still worse, its
application can be nondeterministic.  By distributing the workload to
tone 3 sandhi, its computational nature is effectively obscured.
Moreover, using autosegmental representations to represent
\textemph{hierarchical} structures is in some sense an abuse.

\section{Tianjin Mandarin: Rule Interaction}
Tianjin Mandarin is a more interesting case than Standard Chinese, as
there are a total of \textemph{three} tone sandhi rules that interact
in a nontrivial way.  In Tianjin Mandarin, three dissimilation rules
similar to tone 3 sandhi in Standard Chinese are found:\footnote{For
  convenience, notations borrowed from Hanyu Pinyin are used to
  indicate the tones, although they do not phonologically correspond
  to the specific tones in Standard Chinese.}

\ex.
\a. \(\grave{\sigma}\grave{\sigma} \to \check{\sigma}\grave{\sigma}\)
\hfill(F[alling-]F sandhi)
\b. \(\check{\sigma}\check{\sigma} \to \acute{\sigma}\check{\sigma}\)
\hfill(L[ow-]L sandhi)
\b. \(\acute{\sigma}\acute{\sigma} \to \bar{\sigma}\acute{\sigma}\)
\hfill(R[ising-]R sandhi)

These rules approximately generalize to trisyllabic sequences in a
manner reminiscent of directionalities:

\ex.
\a. \(\grave{\sigma}\grave{\sigma}\grave{\sigma} \to
\grave{\sigma}\check{\sigma}\grave{\sigma}\)
\hfill(right-to-left)
\b. \(\check{\sigma}\check{\sigma}\check{\sigma} \to
\check{\sigma}\acute{\sigma}\check{\sigma}\)
\hfill(right-to-left)
\b. \(\acute{\sigma}\acute{\sigma}\acute{\sigma} \to
\bar{\sigma}\acute{\sigma}\acute{\sigma} \to
\bar{\sigma}\bar{\sigma}\acute{\sigma}\)
\hfill(left-to-right)


There has been debates on how these rules interact and, more
importantly, how to explain such interactions.  Rule-based analyses
such as \posscite{t87tstd, z87ptal} proposals rely on stipulations to
account for the directionalities as well as the feeding relations,
therefore offering no real explanation.  In the subregular zoo,
instead of directionalities, the difference in patterns amounts to the
difference in logical formulae, that is, the computational nature of
the rules.  The problem is then how to also accommodate the feeding
relations.  A relevant \enquote{boundaryless} computational analysis
will be discussed before a new \enquote{boundaryful} analysis is
presented.

\subsection{Bounadaryless Analysis}
\posscite{c18catsi} analysis is based on the observation that the
three strictly local functions can be combined into a single one,
which must remember a pair of input and output substrings.  It is thus
an input--output strictly local function.  Following our convention so
far, the idea will be described using logical formulae instead of
transducers.  The logical transduction, without consideration of the
feeding relations, is as follows:

\ex.
\a. \(\check{\sigma}'(x) \coloneq
(\check{\sigma}(x) \land \neg\check{\sigma}'(\mathit{succ}(x))) \lor
(\grave{\sigma}(x) \land \grave{\sigma}'(\mathit{succ}(x)))\)
\b. \(\acute{\sigma}'(x) \coloneq
(\acute{\sigma}(x) \land \neg\acute{\sigma}(\mathit{succ}(x))) \lor
(\check{\sigma}(x) \land \check{\sigma}'(\mathit{succ}(x)))\)
\b. \(\bar{\sigma}'(x) \coloneq \bar{\sigma}(x) \lor
(\acute{\sigma}(x) \land \acute{\sigma}(\mathit{succ}(x)))\)

In the transduction, rule interactions are naturally derived from the
shapes of logical formulae.  We see that FF sandhi feeds LL sandhi
given the presence of \(\check{\sigma}'\), and similarly that LL
sandhi counterfeeds RR sandhi given the absence of
\(\acute{\sigma}'\).  A redefinition of the transduction gives us
another possibility:

\ex.
\a. \(\check{\sigma}'(x) \coloneq
(\check{\sigma}(x) \land
\neg\check{\sigma}^{\land}(\mathit{succ}(x))) \lor
(\grave{\sigma}(x) \land \grave{\sigma}'(\mathit{succ}(x)))\)
\b. \(\acute{\sigma}'(x) \coloneq
(\acute{\sigma}(x) \land
\neg\acute{\sigma}^{\lor}(\mathit{succ}(x))) \lor
(\check{\sigma}(x) \land \check{\sigma}^{\land}(\mathit{succ}(x)))\)
\b. \(\bar{\sigma}'(x) \coloneq \bar{\sigma}(x) \lor
(\acute{\sigma}(x) \land
\acute{\sigma}^{\lor}(\mathit{succ}(x)))\)
\b. \(\check{\sigma}^{\land}(x) \coloneq
\check{\sigma}(x) \land \check{\sigma}'(x)\)
\b. \(\acute{\sigma}^{\lor}(x) \coloneq
\acute{\sigma}(x) \lor \acute{\sigma}'(x)\)

FF sandhi now counterfeeds LL sandhi given the presence of
\(\check{\sigma}^{\land}\), that is, a sandhi low tone does
\textemph{not} trigger LL sandhi.  Likewise, LL sandhi now feeds RR
sandhi given the presence of \(\acute{\sigma}^{\lor}\), that is, a
rising tone either in the input or output sequence triggers RR sandhi.


This analysis apparently does not make reference to boundaries, unlike
our analysis of tone 3 sandhi in Standard Chinese.  This leads a
difficulty in describing the conflicting data:

\ex. \(\check{\sigma}\grave{\sigma}\grave{\sigma} \to
\{\check{\sigma}\check{\sigma}\grave{\sigma},
\acute{\sigma}\check{\sigma}\grave{\sigma}\}\)

The former alternative suggests that FF sandhi counterfeeds LL sandhi,
while the latter suggests otherwise:

\ex.
\a. \(\check{\sigma}\grave{\sigma}\grave{\sigma} \to
\check{\sigma}\check{\sigma}\grave{\sigma}\)
\b. \(\check{\sigma}\grave{\sigma}\grave{\sigma} \to
\check{\sigma}\check{\sigma}\grave{\sigma} \to
\acute{\sigma}\check{\sigma}\grave{\sigma}\)

The presence of the extra step of application is exactly what
differentiates counterfeeding and feeding, at least given the lack of
boundaries.  Unfortunately, we can encode either counterfeeding or
feeding, but not both.  The contrast of the two, nonetheless, is
analogous to the situation in Standard Chinese.  Indeed, the following
holds assuming FF sandhi feeds LL sandhi:

\ex.
\a. \(\tau(\check{\sigma}) \cdot \tau(\grave{\sigma}\grave{\sigma}) =
\check{\sigma}\check{\sigma}\grave{\sigma}\)
\b. \(\tau(\check{\sigma}\grave{\sigma}\grave{\sigma}) =
\acute{\sigma}\check{\sigma}\grave{\sigma}\)

\subsection{Boundaryful Analysis}
The data shown in \textcite{t87tstd} are actually subject to a clear
contrast between left-branching and right-branching structures, which
we now know derive different boundaries.  The most trivial case of
boundaries is found when every syllable is emphatically stressed, thus
creating the sequence

\ex. \(v\#w\#z\)

that does not undergo any sandhi.  This happens since a single
syllable can not be affected by the transduction, namely that

\ex. \(\tau(v) \cdot \tau(w) \cdot \tau(z) = vwz\)

This is not particularly interesting, but does show that boundaries
are in effect.  This case is henceforth ignored for lack of interest.

The more interesting cases are found when some syllables do undergo
sandhi.  In these cases, Tianjin Mandarin uses a more
\enquote{conservative} approach to creating boundaries, such that
every constituent receives a prefix and suffix boundary when
flattened.  This leads to the tree transformation

\ex.
\a. \(
\begin{forest}
  nice empty nodes,
  [[\(v\), baseline] [[\(w\)] [\(z\)]]]
\end{forest}
\to
\#v\#wz\#\#\)
\b. \(
\begin{forest}
  nice empty nodes,
  [[[\(v\)] [\(w\)]] [\(z\), baseline]]
\end{forest}
\to
\#\#vw\#z\#\)

that appears as if every bracket is rewritten into a boundary term.
In other words, when boundaries are outputted at all, a left-branching
structure causes the substring of first two syllables to undergo tone
sandhi, while a right-branching one does that of last two.  As usual,
the tree transformation is nondeterministic so that boundaries are
allowed to be omitted.

With this in mind, it is still true that the patterns that are
attested for \textemph{both} left-branching and right-branching
structures can be used to justify the trisyllabic generalization, as
these patterns occur when boundaries are omitted.  The transduction
therefore remains strictly local, while the originally conflicting
data can now be well explained:

\ex. \(\check{\sigma}(\grave{\sigma}\grave{\sigma}) \to
\{\check{\sigma}\#\check{\sigma}\grave{\sigma},
\acute{\sigma}\check{\sigma}\grave{\sigma}\}\)


Applying this idea to more data shows that it is indeed the right
track.  The cases of trisyllabic sequences are shown first, as these
cases do not involve the complication of feeding relations:

\ex.
\a. \(\check{\sigma}(\check{\sigma}\check{\sigma}) \to
\{\check{\sigma}\#\acute{\sigma}\check{\sigma},
\check{\sigma}\acute{\sigma}\check{\sigma}\}\)
\b. \((\check{\sigma}\check{\sigma})\check{\sigma} \to
\{\acute{\sigma}\check{\sigma}\#\check{\sigma},
\check{\sigma}\acute{\sigma}\check{\sigma}\}\)
\b. \(\acute{\sigma}(\acute{\sigma}\acute{\sigma}) \to
\{\acute{\sigma}\#\bar{\sigma}\acute{\sigma},
\bar{\sigma}\bar{\sigma}\acute{\sigma}\}\)
\b. \((\acute{\sigma}\acute{\sigma})\acute{\sigma} \to
\{\bar{\sigma}\acute{\sigma}\#\acute{\sigma},
\bar{\sigma}\bar{\sigma}\acute{\sigma}\}\)

The FF sandhi data also suggest a minor modification to the
transduction to accommodate the rule

\ex. \(\grave{\sigma}\check{\sigma} \to \bar{\sigma}\check{\sigma}\)
\hfill(FL sandhi)

that produces the following mappings:

\ex.
\a. \(\grave{\sigma}\grave{\sigma}\grave{\sigma} \to
\bar{\sigma}\check{\sigma}\grave{\sigma}\)
\b. \(\grave{\sigma}\grave{\sigma}\check{\sigma} \to
\check{\sigma}\bar{\sigma}\check{\sigma}\)

This can be done by redefining the logical transduction:

\ex.
\a. \(\check{\sigma}'(x) \coloneq
(\check{\sigma}(x) \land
\neg\check{\sigma}^{\land}(\mathit{succ}(x))) \lor
(\grave{\sigma}(x) \land
\grave{\sigma}(\mathit{succ}(x)) \land
\neg\check{\sigma}'(\mathit{succ}(x)))\)
\b. \(\bar{\sigma}'(x) \coloneq \bar{\sigma}(x) \lor
(\acute{\sigma}(x) \land
\acute{\sigma}^{\lor}(\mathit{succ}(x))) \lor
(\grave{\sigma}(x) \land
\check{\sigma}'(\mathit{succ}(x)))\)

The interaction is a curious one, as now FF sandhi feeds FL sandhi
\textemph{while} counterbled by it.  This is rather complicated to
describe with cyclicity, but simple enough as a strictly local
function.  Correct patterns of FF sandhi can now be predicted:

\ex.
\a. \(\grave{\sigma}(\grave{\sigma}\grave{\sigma}) \to
\{\grave{\sigma}\#\check{\sigma}\grave{\sigma},
\bar{\sigma}\check{\sigma}\grave{\sigma}\}\)
\b. \((\grave{\sigma}\grave{\sigma})\grave{\sigma} \to
\{\check{\sigma}\grave{\sigma}\#\grave{\sigma},
\bar{\sigma}\check{\sigma}\grave{\sigma}\}\)

To decide the feeding relations, data involving rule interactions are
examined:

\ex.
\a. LL sandhi bleeds FL sandhi
\a. \(\grave{\sigma}(\check{\sigma}\check{\sigma}) \to
\{\grave{\sigma}\#\acute{\sigma}\check{\sigma},
\grave{\sigma}\acute{\sigma}\check{\sigma}\}\)
\b. \((\grave{\sigma}\check{\sigma})\check{\sigma} \to
\{\bar{\sigma}\check{\sigma}\#\check{\sigma},
\grave{\sigma}\acute{\sigma}\check{\sigma}\}\)
%
\z.
\b. FL sandhi counterbleeds FF sandhi
\a. \(\grave{\sigma}(\grave{\sigma}\check{\sigma}) \to
\{\grave{\sigma}\#\bar{\sigma}\check{\sigma},
\check{\sigma}\bar{\sigma}\check{\sigma}\}\)
\b. \((\grave{\sigma}\grave{\sigma})\check{\sigma} \to
\{\check{\sigma}\grave{\sigma}\#\check{\sigma},
\check{\sigma}\bar{\sigma}\check{\sigma}\}\)
%
\z.
\b. LL sandhi feeds RR sandhi
\a. \(\acute{\sigma}(\check{\sigma}\check{\sigma}) \to
\{\acute{\sigma}\#\acute{\sigma}\check{\sigma},
\bar{\sigma}\acute{\sigma}\check{\sigma}\}\)
\b. \((\acute{\sigma}\check{\sigma})\check{\sigma} \to
\{\acute{\sigma}\check{\sigma}\#\check{\sigma},
\bar{\sigma}\acute{\sigma}\check{\sigma}\}\)
%
\z.
\b. FF sandhi feeds LL sandhi
\a. \(\check{\sigma}(\grave{\sigma}\grave{\sigma}) \to
\{\check{\sigma}\#\check{\sigma}\grave{\sigma},
\acute{\sigma}\check{\sigma}\grave{\sigma}\}\)
\b. \((\check{\sigma}\grave{\sigma})\grave{\sigma} \to
\{\check{\sigma}\grave{\sigma}\#\grave{\sigma},
\acute{\sigma}\check{\sigma}\grave{\sigma}\}\)

All that remains is a transduction that is compatible with all cases,
which can be defined using the shown techniques.


Lastly, some rare cases in the data seem to suggest that LL sandhi
produces the alternative mapping:

\ex. \(\check{\sigma}\check{\sigma}\check{\sigma} \to
\acute{\sigma}\acute{\sigma}\check{\sigma}\)

As previously discussed, adding alternatives to the transduction
incurs nondeterminism.  Instead, it is more likely that some lexical
items with a low tone have alternative realizations with a rising
tone, moving nondeterminism out from the transduction.\footnote{This
  can also explain the mapping
  \(\check{\sigma}\grave{\sigma}\grave{\sigma} \to
  \acute{\sigma}\check{\sigma}\grave{\sigma}\) even if counterfeeding
  is assumed.}

\section{Interim Summary}
This section has argued for a modular analysis of tonal transformation
that is sensitive to the hierarchical structure.  The point is not to
suggest that a certain analysis is inherently better than another.
Rather, it argues that such a modular analysis is computationally
feasible and clarifies the boundaries between linguistic levels.  Even
if the overall complexity of the composed system remains the same, a
modular analysis allows for the local reasoning of each module due to
abstraction and encapsulation.

Specifically, a boundary-based analysis is presented for both Standard
Chinese and Tianjin Mandarin, where a boundary is merely a term that
blocks the transformation.  In this sense, the whole transformation
can be understood as a boundary-agnostic subtransformation
\enquote{lifted} to operate on multiple linear sequences that are
produced by splitting the input sequence by boundaries.  This is how we
achieve modularity---by further and further separating out each step
of computation and composing them together at the end.

The theoretical implication is that the methodological assumption of
lexical phonology or any modular approach in general is preferable.
Nonetheless, this does not and should not bear any substantive
implication.  Nothing is assumed about the \enquote{architecture of
  the language faculty}, as this issue is simply not important in the
computational understanding of phonology.  We agree with lexical
phonology insomuch as both are opposed to the cyclic and thus
monolithic view of phonology.  Anything more is left aside.

\chapter{Power of Representation}
In the previous chapter, we have had a taste of how representations
can obscure the computational nature.  This chapter relates this
intuition to even more attested tonal transformations.
Representations are indispensable both as a theoretical tool and
metatheoretical notion, and they often enforce specific ways to
express the computation.  Choosing an incorrect representation can
either overly simplify or complicate the computation, much to our
dismay.

\section{Representation as Computation}
Representations, when considered as a part of bigger computation,
implicitly \textterm{encode} partially or fully done computation.
This is not new to linguists, as syntacticians often face the
dichotomy between features and constraints, whose equivalence is
pointed out in \textcite{g17cgdfc}.  The contentious issue of enriched
representations has also been a long-standing problem in theoretical
phonology.

Most significantly, choosing an unintuitive representation can lead to
a seemingly simplified computation, at the expense of a more complex
conversion from the intuitive representation.  A famous example is how
\posscite{d72lcnndtafmact} notation highly simplifies substitution in
\(\lambda\)-calculi by eliminating variable shadowing, despite the
fact that the conversion from a conventional notation requires taking
care of exactly that!  As another example, recall the unattested
transformation that rewrites \(a\) depending on the evenness of its
number.  Suppose the representation encodes this information by
splitting \(a\) into \(\acute{a}\) and \(\grave{a}\), with the
following constraints:

\ex.
\a. \(\acute{a}(x) \implies (\forall y\ldotp x < y \implies
\neg\grave{a}(y))\)
\b. \((\acute{a}(x) \land
(\neg\exists y\ldotp x \vartriangleleft_{a} y))
\implies \mathit{Odd}(x)\)
\b. \(\grave{a}(x) \implies (\forall y\ldotp x < y \implies
\neg\acute{a}(y))\)
\b. \((\grave{a}(x) \land
(\neg\exists y\ldotp x \vartriangleleft_{a} y))
\implies \mathit{Even}(x)\)

The transformation then simply rewrites each \(\acute{a}\) to \(a\)
and \(\grave{a}\) to \(b\).  All is well, except now the
representation not only encodes the evenness information, but also
renders many sequences invalid.  It is only reasonable when it serves
as an \textemph{intermediate} representation that encodes the decision
of the evenness.  Even then, the decision would have had the same
computational complexity as before anyway.

More surprisingly, representations can also complicate the
computation, in particular when the structures are unsuited for the
problem domain.  The familiar issue of cyclicity is precisely one
example, as it essentially leads to a hierarchical structure in
phonology.  This hierarchical structure, in turn, creates the
important question of conditions of application, which is one of the
reasons regular models were adopted in \textcite{kk94rmprs}.
Conversely, the fact that arbitrary term rewriting generates
recursively enumerable languages as proved by \textcite{pr73gptg}
suggests that a more principled \enquote{movement} theory is required,
ideally based on hierarchical structures.  Nowadays, variants of
categorial grammars such as combinatory categorial grammars in
\textcite{s00sp} are adopted.

The mention of categorial grammars relates to a fundamental issue in
our metatheory.  Our metatheory is inherently a logical one, partially
due to its ease of understanding, but more so due to the equational
reasoning provided by logical calculi.  A \textterm{calculus}, in
essence, is a theory of the equivalence between representations,
defined \enquote{up to} some relations.  By studying the calculus
directly or indirectly, we gain a better understanding of the
essential properties in the formalisms, in contrast to idiosyncratic
artifacts of our metalanguage.  For example, \(\lambda\)-terms that
differ only in bound variable names are equivalent up to
\(\alpha\)-equivalance, suggesting that names should not matter.  In
the case of subregular functions, the logical transductions are their
defining property, as opposed to the theoretically inspired
representations.

\section{Shanghai Wu: Tone Spreading}
Shanghai Wu has a tonal transformation traditionally known as
\enquote{left dominance}, properly named so after the fact that the
leftmost syllable decides the tonal pattern of the whole group.
Detailed theoretical discussions can be found in \textcite{d93rlsad,
  d99mstems} among others.  This tonal transformation is particularly
relevant as it is often analyzed as an \textemph{autosegmental}
phenomenon, despite its almost linear nature.

Intuitively, the idea is that the first term on the tonal tier is
always associated with the first syllable, while the second term is
associated with either the first or second syllable depending on
whether the latter is present.  The remaining syllables, if any,
remain unassociated with any term and are thus \enquote{toneless}.
This gives us the following mappings:

\ex.
\a. \(
\begin{forest}
  auto, [\(\sigma\) [T, no edge] [T, no edge]]
\end{forest}
\to
\begin{forest}
  auto, [\(\sigma\) [T] [T]]
\end{forest}\)
\b. \(
\begin{forest}
  auto,
  [, phantom
  [\(\sigma\), baseline [T, no edge]]
  [\(\sigma\) [T, no edge]]
  [\(\sigma\ldots\) [\(\text{T}\ldots\), no edge]]]
\end{forest}
\to
\begin{forest}
  auto,
  [, phantom
  [\(\sigma\), baseline [T]]
  [\(\sigma\) [T]]
  [\(\sigma\ldots\) [\(\text{T}\ldots\), no edge]]]
\end{forest}\)


It is worth it to ask whether autosegmental representations are really
needed.  For example, it is not apparent why the excessive terms on
the tonal tier should be present at all if they always end up unused
by the transformation.  A bigger problem is that the tonal status of a
specific syllable is always tied to the association relation, forcing
us to retreat to this unnecessary indirection and obscuring the nature
of left dominance.

To demonstrate the effects of this theoretical assumption of
autosegmental representations, an autosegmental analysis proposed by
\textcite{z20eaislfmdft} will be discussed, followed by our linear
analysis.  As we will see, there is little to no reason to assume
autosegmental representations, at least in the case of Shanghai Wu.
It is in some sense overly generalized, contra the spirit of pinning
down the exact computational complexity, as well as creating the
problem of normalized representations.

\subsection{Autosegmental Analysis}
\posscite{z20eaislfmdft} analysis is a \enquote{port} of the
theoretical analysis to the computational land.  It is argued that
left dominance \textemph{cannot} be a linear transformation based on
the assumption that a contour tone consists of multiple tonal terms.
Indeed, a hypothetical linear transformation involving \(n\) tonal
terms seems to require a memory size of \(n\), as

\ex. \(\text{T}_{1}\text{T}_{2}\ldots\text{T}_{n-1}.\text{T}_{n} \to
\text{T}_{1}.\text{T}_{2}\)

must remember \(\text{T}_{2}\) up to \(\text{T}_{n}\), crossing the
syllabic boundary.  This assumption, of course, deserves further
questioning.  Regardless, the proposed analysis defines the
transduction as

\ex.
\(\mathit{Assoc}'(x, y) \coloneq
\begin{aligned}[t]
  & (\mathit{First}(x) \land \mathit{First}(y) \land
    \mathit{Assoc}(x, y)) \lor \mbox{}\\
  &\shift (\mathit{Second}(x) \land \mathit{Second}(y) \land
    \mathit{Assoc}(x, \mathit{pred}(y)))
\end{aligned}\)

where \(\mathit{First}\) and \(\mathit{Second}\) are predicates that
test whether the term is the \enquote{first} or \enquote{second} term.
They can be defined as follows assuming left dominance applies to
\(\#\)-separated units:

\ex.
\a. \(\mathit{First}(x) \coloneq \#(\mathit{pred}(x))\)
\b. \(\mathit{Second}(x) \coloneq
\neg\#(\mathit{pred}(x)) \land \#(\mathit{pred}(\mathit{pred}(x)))\)

Similarly, a predicate \(\mathit{Last}\) that tests whether the term
is the \enquote{last} term can be defined, which will become useful
later:

\ex. \(\mathit{Last}(x) \coloneq \#(\mathit{succ}(x))\)


This transduction does not generate the exact mapping we are looking
for.  Instead, it generates

\ex. \(
\begin{forest}
  auto,
  [, phantom
  [\(\sigma\), baseline, name=sfst [T]]
  [\(\sigma\) [T, name=tsnd, no edge]]
  [\(\sigma\ldots\) [\(\text{T}\ldots\), no edge]]]
  \draw (tsnd.north) -- (sfst.south);
\end{forest}
\to
\begin{forest}
  auto,
  [, phantom
  [\(\sigma\), baseline [T]]
  [\(\sigma\) [T]]
  [\(\sigma\ldots\) [\(\text{T}\ldots\), no edge]]]
\end{forest}\)

This imposes an undesirable assumption of the \enquote{underlying}
association, even though it never surfaces in this specific mapping.
Supposedly, the assumption is made to account for the trivial case of
monosyllabic sequences, where the mapping is

\ex. \(
\begin{forest}
  auto,
  [\(\sigma\) [T] [T]]
\end{forest}
\to
\begin{forest}
  auto,
  [\(\sigma\) [T] [T]]
\end{forest}\)

that is, an identity map.  The transduction, however, does not reflect
this intuition of \enquote{citation} forms, as

\ex. \(
\begin{forest}
  auto,
  [, phantom
  [\(\sigma\), baseline, name=sfst [T]]
  [\phantom{\(\sigma\)} [T, name=tsnd, no edge]]
  [\phantom{\(\sigma\)} [\(\text{T}\ldots\), no edge]]]
  \draw (tsnd.north) -- (sfst.south);
\end{forest}
\to
\begin{forest}
  auto,
  [, phantom
  [\(\sigma\), baseline, name=sfst [T]]
  [\phantom{\(\sigma\)} [T, name=tsnd, no edge]]
  [\phantom{\(\sigma\)} [\(\text{T}\ldots\), no edge]]]
\end{forest}\)

does not preserve the underlying association.

Upon further examination, \(\mathit{Assoc}\) appears to be unnecessary
in the transduction.  A redefinition of the transduction can drop the
reference to it, as the surface association merely depends on the
\textemph{positions} of the terms:

\ex.
\a. \(\mathit{Assoc}'(x, y) \coloneq
(\mathit{First}(x) \land \mathit{First}(y)) \lor
(\mathit{Second}(x) \land
(\mathit{Only}(y) \lor \mathit{Second}(y)))\)
\b. \(\mathit{Only}(x) \coloneq
\mathit{First}(x) \land \mathit{Last}(x)\)

This transduction directly corresponds to the theoretical analysis and
assumes nothing about the underlying association.  This gives rise to
the question whether the \enquote{garbage} tonal terms are retained in
the surface forms---if a normalization procedure is in effect.
Intuitively, the surface forms should always be normalized to encode
the \textemph{least} information as possible.  Such normalization can
be achieved by tailoring the domain predicate that decides whether or
not a term is copied:\footnote{And, accordingly, the predecessor and
  successor functions are adjusted to the domain.}

\ex. \(\mathit{Copy}(x) \coloneq \neg\text{T}(x) \lor
\mathit{First}(x) \lor \mathit{Second}(x)\)


A more severe problem is that this transformation misses the
\textform{yángrù} pattern.  This pattern occurs when the first
syllable both starts with a voiced onset (\enquote{\textform{yáng}})
and ends with a stop coda (\enquote{\textform{rù}}), and is much
different from the above pattern:

\ex. \(
\begin{forest}
  auto,
  [, phantom
  [\(\sigma\), baseline [T, no edge]]
  [\(\sigma\ldots\)]
  [\(\sigma\) [T, no edge]]
  [\phantom{\(\sigma\)} [\(\text{T}\ldots\), no edge]]]
\end{forest}
\to
\begin{forest}
  auto,
  [, phantom
  [\(\sigma\), baseline [T, name=tfst]]
  [\(\sigma\ldots\), name=ssnd]
  [\(\sigma\), name=slst [T]]
  [\phantom{\(\sigma\)} [\(\text{T}\ldots\), no edge]]]
  \draw (tfst.north) -- (ssnd.south west);
  \draw (tfst.north) -- (slst.south);
\end{forest}\)

This transformation requires a recursive definition:

\ex. \(
\mathit{Assoc}'(x, y) \coloneq
(\mathit{First}(x) \land \mathit{First}(y)) \lor \mbox{}
\mathit{Assoc}'(x, \mathit{pred}(y)) \lor \mbox{}
(\mathit{Second}(x) \land \mathit{Last}(y))\)

Effectively, it spreads the first tonal term to all syllables and
associates the second tonal term to the last syllable.

The combination of the two transformations is more involved, but not
impossible.\footnote{Alternatively, the \textform{yángrù} pattern is
  simply subject to a separate transformation.  This is not
  implausible given that some speakers only use the pattern for
  disyllabic and trisyllabic sequences.} Observe the following
implication in all surface patterns:

\ex. \((\mathit{Second}(x) \land
\mathit{Last}(y) \land
\mathit{Assoc}'(\mathit{pred}(x), y)) \implies
\mathit{Assoc}'(x, y)\)

This implication subsumes the use of \(\mathit{Only}\), since

\ex. \(\mathit{Only}(x)
\implies \mathit{First}(x)
\implies \exists y\ldotp
(\mathit{First}(y) \land \mathit{Assoc}'(y, x))\)

Moreover, a way to limit the spreading to the \textform{yángrù}
pattern is needed.  This can be simply done by splitting the spreading
into two parts, where the first part associates the first tonal term
to the second syllable, and the second part further spreads the first
tonal term.  This way, a check can be added to both parts.  Therefore,
the combined transformation is

\ex. \(
\begin{aligned}[t]
  &\mathit{Assoc}'(x, y) \coloneq \mbox{}\\
  &\shift (\mathit{First}(x) \land \mathit{First}(y)) \lor \mbox{}\\
  &\shift (\mathit{First}(x) \land
    \mathit{Second}(y) \land
    \mathit{Yángrù}(\mathit{pred}(y))) \lor \mbox{}\\
  &\shift (\mathit{Assoc}'(x, \mathit{pred}(y)) \land
    \mathit{Assoc}'(x, \mathit{pred}(\mathit{pred}(y)))) \lor
    \mbox{}\\
  &\shift (\mathit{Second}(x) \land \mbox{}\\
  &\shift\shift ((\mathit{Second}(y) \land
    \neg\mathit{Yángrù}(\mathit{pred}(y))) \lor
    (\mathit{Last}(y) \land \mathit{Assoc}'(\mathit{pred}(x), y))))
\end{aligned}\)

\subsection{Linear Analysis}
At this point, it is worth it to revisit the linear transformation.
As we notice, there are only \textemph{two} possible configurations in
the non-\textform{yángrù} patterns.  This results in a correspondence
to the previously assumed linear representations:

\ex.
\a. \(
\begin{forest}
  auto,
  [\(\sigma\) [T] [T]]
\end{forest}
\implies
\text{TT}\)
\b. \(
\begin{forest}
  auto,
  [, phantom
  [\(\sigma\), baseline [T]]
  [\(\sigma\) [T]]
  [\(\sigma\ldots\)]]
\end{forest}
\implies
\text{T}.\text{T}\)

In this sense, syllabic boundaries are \textemph{not} needed in the
inputs at all, as they can be created by another transformation:

\ex.
\a. \(.'(x) \coloneq \mathit{Second}(x) \land \neg\mathit{Last}(x)\)
\b. \(\text{T}'(x) \coloneq \mathit{First}(x) \lor
\mathit{Last}(x) \lor
\mathit{Second}(\mathit{pred}(x))\)

Unfortunately, the \textform{yángrù} pattern does not show an apparent
correspondence to linear representations of this kind.  The major
problem is that the use of syllabic boundary above is not for
delimiting units, but rather as a \enquote{diacritic} to mark the
alternative configuration.  This falls down for the \textform{yángrù}
pattern, as there are now \(n\) configurations where
\(n \in \mathbb{N}\), echoing with the original argument against
linear transformation.

Alternatively, we can resort to a strictly one-to-one association
between syllables and tones, that is, marking the tones on the
syllables.  In fact, the tonal patterns in Shanghai Wu are limited to
these four types, where the first two types are accounted for by the
classical analysis of left dominance:\footnote{Tones are indicated as
  in the International Phonetic Alphabet.}

\ex.
\a. \(\{\hat{\sigma}, \acute{\sigma}\grave{\sigma}\sigma\ldots\}\)
\hfill(\textform{yīnpíng})
\b. \(\{\check{\sigma},
\grave{\sigma}\acute{\sigma}\sigma\ldots\}\)
\hfill(\textform{shǎngqùshēng})
\b. \(\{\acute{\sigma},
\grave{\sigma}\acute{\sigma}\sigma\ldots\}\)
\hfill(\textform{yīnrù})
\b. \(\{\check{\sigma},
\grave{\sigma}\grave{\sigma}\ldots\check{\sigma}\}\)
\hfill(\textform{yángrù})

First, consider a linear transformation that accounts for the
non-\textform{yángrù} patterns.  Following the intuition that the
monosyllabic sequences reflect some sort of \enquote{citation} forms,
the tones are left unchanged when they are on the only syllables:

\ex.
\a. \(\hat{\sigma}'(x) \coloneq
\mathit{Only}(x) \land \hat{\sigma}(x)\)
\b. \(\check{\sigma}'(x) \coloneq
\mathit{Only}(x) \land \check{\sigma}(x)\)

In polysyllabic cases, the first two syllables have their tones
decided by the first syllable:

\ex.
\a. \(
\begin{aligned}[t]
  & \acute{\sigma}'(x) \coloneq \mbox{}\\
  &\shift (\mathit{Only}(x) \land \acute{\sigma}(x)) \lor \mbox{}\\
  &\shift (\mathit{FirstNonLast}(x) \land
    \hat{\sigma}(x)) \lor \mbox{}\\
  &\shift (\mathit{Second}(x) \land
    (\check{\sigma}(\mathit{pred}(x)) \lor
    \acute{\sigma}(\mathit{pred}(x))))
\end{aligned}\)
\b. \(\grave{\sigma}'(x) \coloneq
(\mathit{FirstNonLast}(x) \land
(\check{\sigma}(x) \lor \acute{\sigma}(x))) \lor
(\mathit{Second}(x) \land \hat{\sigma}(\mathit{pred}(x)))\)
\b. \(\mathit{FirstNonLast}(x) \coloneq
\mathit{First}(x) \land \neg\mathit{Last}(x)\)

The remaining syllables are assigned neutral tones:

\ex. \(\sigma'(x) \coloneq
\neg(\mathit{First}(x) \lor \mathit{Second}(x))\)


The above transformation, like its autosegmental counterpart, is
nonrecursive.  However, the \textform{yángrù} pattern still requires a
recursive definition given that it is characteristically a tone
spreading rule:

\ex.
\a. \(\grave{\sigma}'(x) \coloneq
\mathit{FirstNonLast}(x) \lor
(\neg\mathit{Last}(x) \land \grave{\sigma}'(\mathit{pred}(x)))\)
\b. \(\check{\sigma}'(x) \coloneq \mathit{Last}(x)\)

The combination of the two transformations, again, requires the
distinction between the \textform{yángrù} and non-\textform{yángrù}
patterns.  The same technique can be adapted to the linear
transformation, with the caveat that tone spreading applies up to the
\textemph{penultimate} syllable.  The resulting transduction is as
follows, omitting predicates that carry their original
definitions:\footnote{For convenience, \textform{yángrù} is assumed to
  be a unique tone in the inputs, although such an assumption does not
  otherwise change the overall complexity, as \textform{yángrù} can be
  recognized by the segmental features.}

\ex.
\a. \(
\begin{aligned}[t]
  & \check{\sigma}'(x) \coloneq \mbox{}\\
  &\shift (\mathit{Only}(x) \land
    (\check{\sigma}(x) \lor \mathit{Yángrù}(x))) \lor \mbox{}\\
  &\shift (\mathit{Last}(x) \land \mbox{}\\
  &\shift\shift ((\mathit{Second}(x) \land
    \mathit{Yángrù}(\mathit{pred}(x))) \lor
    (\grave{\sigma}'(\mathit{pred}(x)) \land
    \grave{\sigma}'(\mathit{pred}(\mathit{pred}(x))))))
\end{aligned}\)
\b. \(
\begin{aligned}[t]
  & \grave{\sigma}'(x) \coloneq \mbox{}\\
  &\shift (\mathit{FirstNonLast}(x) \land
    (\check{\sigma}(x) \lor
    \acute{\sigma}(x) \lor
    \mathit{Yángrù}(x))) \lor \mbox{}\\
  &\shift (\mathit{Second}(x) \land
    \hat{\sigma}(\mathit{pred}(x))) \lor \mbox{}\\
  &\shift (\neg\mathit{Last}(x) \land \mbox{}\\
  &\shift\shift ((\mathit{Second}(x) \land
    \mathit{Yángrù}(\mathit{pred}(x))) \lor
    (\grave{\sigma}'(\mathit{pred}(x)) \land
    \grave{\sigma}'(\mathit{pred}(\mathit{pred}(x))))))
\end{aligned}\)
\b. \(\sigma'(x) \coloneq \neg(
\mathit{First}(x) \lor
\mathit{Second}(x) \lor
(\grave{\sigma}'(\mathit{pred}(x)) \land
\grave{\sigma}'(\mathit{pred}(\mathit{pred}(x)))))\)

The last predicate reflects the fact that neutral tones only surface
after a \enquote{contour} formed by the first two syllables.

The combined linear transformation surprisingly resembles its
autosegmental counterpart to some degree, except that more concrete
tonal representations are used.  This is not a coincidence, but merely
a result of the general idea of left dominance.  An implementation of
this idea necessarily involves the distinction between
\textform{yángrù} and non-\textform{yángrù}, as well as that between
monosyllabic and polysyllabic.

On the other hand, the autosegmental transformation is an
\textemph{overgeneralization}.  It is in some sense not a tonal
transformation, but an \enquote{association transformation}, since it
does not assume anything about the tonal terms.  There is therefore
nothing stopping, say, \(\grave{\sigma}\grave{\sigma}\sigma\ldots\)
from being a valid surface pattern given \(\text{LLT}\ldots\) on the
tonal tier, while the linear transformation does not acknowledge the
\(\grave{\sigma}\) predicate at all.  Normalization is also not a
problem in the linear transformation, as each tone is strictly marked
on each syllable, whereas the autosegmental transformation does not
care much beyond the association relation.

\section{Suzhou Wu: Exceptional Pattern}
Suzhou Wu is similar to Shanghai Wu in that left dominance patterns
can be found in some heavy-initial sequences, where \enquote{heavy}
refers to syllables that do not end with a stop coda.  Following
\posscite{z20qffselts, z23maltssw} theoretical analysis, there are
five types for such syllables:

\ex.
\a. ti\textsuperscript{HH} \textgls{low}
\hfill(\textform{yīnpíng})
\b. di\textsuperscript{LH} \textgls{carry}
\hfill(\textform{yángpíng})
\b. ti\textsuperscript{HL} \textgls{bottom}
\hfill(\textform{yīnshǎng})
\b. ti\textsuperscript{HLH} \textgls{emperor}
\hfill(\textform{yīnqù})
\b. di\textsuperscript{LHL} \textgls{ground}
\hfill(\textform{yángshǎngqù})

Specifically, \textform{yángpíng}- and \textform{yángshǎngqù}-initial
sequences show an obvious pattern of left dominance as in the
autosegmental mapping:

\ex. \(
\begin{forest}
  auto,
  [, phantom
  [\(\sigma\), baseline [L, no edge]]
  [\(\sigma\) [H, no edge]]
  [\phantom{(}\(\sigma\ldots\) [(L), no edge]]]
\end{forest}
\to
\begin{forest}
  auto,
  [, phantom
  [\(\sigma\), baseline [L]]
  [\(\sigma\) [H]]
  [\phantom{(}\(\sigma\ldots\) [(L), no edge]]]
\end{forest}\)

The same mapping applies to \textform{yīnpíng}-initial sequences
assuming that the mapping is only sensitive to the associations.
However, \textform{yīnshǎng}-initial sequences undergo a different
mapping where the first syllable always bears the same tone:

\ex. \(
\begin{forest}
  auto,
  [, phantom
  [\(\sigma\), baseline [H, no edge]]
  [\(\sigma\) [L, no edge]]
  [\(\sigma\ldots\)]]
\end{forest}
\to
\begin{forest}
  auto,
  [, phantom
  [\(\sigma\), baseline, name=sfst [H]]
  [\(\sigma\) [L, no edge, name=tsnd]]
  [\(\sigma\ldots\)]]
  \draw (tsnd.north) -- (sfst.south);
\end{forest}\)

The second syllable can be assumed to either associate with the second
tonal term or remain unassociated.  Either way, it surfaces with a low
tone.

To account for the contrast between left-dominant and
\enquote{exceptional} patterns, a proposal based on the idea of
\enquote{floating tones} has been introduced.  Basically, the
transformation must now be sensitive to the underlying association
such that only the unassociated tonal terms are subject to the
left-dominant mapping.  By doing so, the transformation is able to
produce both mappings:

\ex.
\a. \(
\begin{forest}
  auto,
  [, phantom
  [\(\sigma\), baseline [T, no edge]]
  [\(\sigma\) [T, no edge]]
  [\(\sigma\ldots\) [\(\text{T}\ldots\), no edge]]]
\end{forest}
\to
\begin{forest}
  auto,
  [, phantom
  [\(\sigma\), baseline [T]]
  [\(\sigma\) [T]]
  [\(\sigma\ldots\) [\(\text{T}\ldots\), no edge]]]
\end{forest}\)
\b. \(
\begin{forest}
  auto,
  [, phantom
  [\(\sigma\), baseline, name=sfst [T]]
  [\(\sigma\) [T, no edge, name=tsnd]]
  [\(\sigma\ldots\) [\(\text{T}\ldots\), no edge]]]
  \draw (tsnd.north) -- (sfst.south);
\end{forest}
\to
\begin{forest}
  auto,
  [, phantom
  [\(\sigma\), baseline, name=sfst [T]]
  [\(\sigma\) [T, no edge, name=tsnd]]
  [\(\sigma\ldots\) [\(\text{T}\ldots\), no edge]]]
  \draw (tsnd.north) -- (sfst.south);
\end{forest}\)

It is therefore natural to ask how such a transformation can be
computationally implemented.  A relevant computational analysis found
in \textcite{z20eaislfmdft} will be discussed and criticized due to
its mischaracterization of the computational complexity, contrasted
with our analysis.  The idea of \enquote{diacritical} features will be
mentioned and justified.

\subsection{Morphemic Analysis}
\posscite{z20eaislfmdft} analysis of Suzhou is a \enquote{morphemic}
analysis in the sense that it involves the addition of an extra
morphemic tier on top of the existing segmental and tonal tiers.  As a
result, \textform{yángpíng} and \textform{yīnshǎng} syllables are
represented as

\ex.
\a.
\begin{forest}
  auto,
  [, phantom
  [\(\sigma\), baseline
  [L, no edge [\(\mathit{Mor}\), name=mfst]]
  [H, no edge, name=tsnd]]]
  \draw (mfst.north) -- (tsnd.south);
\end{forest}
\b.
\begin{forest}
  auto,
  [, phantom
  [\(\sigma\), baseline
  [H [\(\mathit{Mor}\), name=mfst]]
  [L, name=tsnd]]]
  \draw (mfst.north) -- (tsnd.south);
\end{forest}

Curiously, the morphemic association relation
\(\mathit{Rel}_{\mathit{Mor}}\) serves no apparent purpose in the
transduction.  Indeed, the transduction is defined as

\ex. \(
\begin{aligned}[t]
  & \mathit{Assoc}'(x, y) \coloneq \mbox{}\\
  &\shift ((\mathit{Assoc}(x, y) \lor \mathit{First}(x)) \land
    \mathit{First}(y)) \lor \mbox{}\\
  &\shift (\neg\mathit{Assoc}(x, \mathit{pred}(y)) \land
    (\exists z\ldotp
    \mathit{Rel}_{\mathit{Mor}}(z, y) \land \mathit{First}(z)) \land
    \mathit{Second}(x) \land \mathit{Second}(y))
\end{aligned}\)

where \(\mathit{Rel}_{\mathit{Mor}}\) causes the unfortunate presence
of existential quantification.  In \textcite{z20eaislfmdft}, this
existential quantification is left out from the transduction,
resulting in its mischaracterization as strictly local.  This cannot
be true, as the transduction must be logically closed, while the
removal of the existential quantification leads to a free variable.
Regardless, this transduction produces the following mappings:

\ex.
\a. \(
\begin{forest}
  auto,
  [, phantom
  [\(\sigma\), baseline [T, no edge [\(\mathit{Mor}\), name=mfst]]]
  [\(\sigma\) [T, no edge, name=tsnd]]
  [\(\sigma\ldots\) [\(\text{T}\ldots\), no edge]]]
  \draw (mfst.north) -- (tsnd.south);
\end{forest}
\to
\begin{forest}
  auto,
  [, phantom
  [\(\sigma\), baseline [T [\(\mathit{Mor}\), name=mfst]]]
  [\(\sigma\) [T, name=tsnd]]
  [\(\sigma\ldots\) [\(\text{T}\ldots\), no edge]]]
  \draw (mfst.north) -- (tsnd.south);
\end{forest}\)
\b. \(
\begin{forest}
  auto,
  [, phantom
  [\(\sigma\), baseline, name=sfst [T [\(\mathit{Mor}\), name=mfst]]]
  [\(\sigma\) [T, no edge, name=tsnd]]
  [\(\sigma\ldots\) [\(\text{T}\ldots\), no edge]]]
  \draw (mfst.north) -- (tsnd.south);
  \draw (tsnd.north) -- (sfst.south);
\end{forest}
\to
\begin{forest}
  auto,
  [, phantom
  [\(\sigma\), baseline, name=sfst [T [\(\mathit{Mor}\), name=mfst]]]
  [\(\sigma\) [T, no edge, name=tsnd]]
  [\(\sigma\ldots\) [\(\text{T}\ldots\), no edge]]]
  \draw (mfst.north) -- (tsnd.south);
  \draw (tsnd.north) -- (sfst.south);
\end{forest}\)


As the very least, we can consider how to preserve the intuition of
the transduction while removing the unnecessary assumptions.  An
immediate note to make is that the addition of
\(\mathit{Rel}_{\mathit{Mor}}\) contributes nothing to the
transduction whatsoever.  It is introduced purely on a theoretical
basis, much like what we have seen in the case of Shanghai Wu.
Crucially, it disambiguates the cases where a floating tone can be
either from the first or second syllable:

\ex.
\a.
\begin{forest}
  auto,
  [, phantom
  [\(\sigma\), baseline [T [\(\mathit{Mor}\), name=mfst]]]
  [\phantom{\sigma} [T, no edge, name=tsnd]]
  [\(\sigma\) [T [\(\mathit{Mor}\)]]]]]
  \draw (mfst.north) -- (tsnd.south);
\end{forest}
\b.
\begin{forest}
  auto,
  [, phantom
  [\(\sigma\), baseline [T [\(\mathit{Mor}\)]]]
  [\phantom{\sigma} [T, no edge, name=tsnd]]
  [\(\sigma\) [T [\(\mathit{Mor}\), name=msnd]]]]]
  \draw (msnd.north) -- (tsnd.south);
\end{forest}

Nonetheless, there is the possibility that such a distinction is
simply \textemph{unfeasible} given that existential quantification is
needed to make use of the morphemic information.  In fact, it is
rather reminiscent of the problem of grouping previously discussed.
Moreover, this scenario does not appear in Suzhou Wu, not even when
\enquote{light}-initial sequences are concerned.

Another problem is that the transduction is forced to assume that the
relevant tonal terms for exceptional patterns must always be
underlyingly associated with the first syllable.  This assumption is
acceptable as a phonotactic condition, but it does raise the question
whether it can be dropped.  Overall, the modified transduction is
defined as

\ex. \(
\begin{aligned}[t]
  & \mathit{Assoc}'(x, y) \coloneq \mbox{}\\
  &\shift ((\mathit{Assoc}(x, y) \lor \mathit{First}(x)) \land
    \mathit{First}(y)) \lor \mbox{}\\
  &\shift (\neg\mathit{Assoc}(x, \mathit{pred}(y)) \land
    \mathit{Second}(x) \land \mathit{Second}(y))
\end{aligned}\)

with the mere removal of \(\mathit{Rel}_{\mathit{Mor}}\).  Further
adjustments for monosyllabic patterns are omitted, but they are easy
enough to figure out with a similar technique to that used for
Shanghai Wu.  This produces the very same mappings except without the
morphemic tier:

\ex.
\a. \(
\begin{forest}
  auto,
  [, phantom
  [\(\sigma\), baseline [T, no edge]]
  [\(\sigma\) [T, no edge, name=tsnd]]
  [\(\sigma\ldots\) [\(\text{T}\ldots\), no edge]]]
\end{forest}
\to
\begin{forest}
  auto,
  [, phantom
  [\(\sigma\), baseline [T]]
  [\(\sigma\) [T, name=tsnd]]
  [\(\sigma\ldots\) [\(\text{T}\ldots\), no edge]]]
\end{forest}\)
\b. \(
\begin{forest}
  auto,
  [, phantom
  [\(\sigma\), baseline, name=sfst [T]]
  [\(\sigma\) [T, no edge, name=tsnd]]
  [\(\sigma\ldots\) [\(\text{T}\ldots\), no edge]]]
  \draw (tsnd.north) -- (sfst.south);
\end{forest}
\to
\begin{forest}
  auto,
  [, phantom
  [\(\sigma\), baseline, name=sfst [T]]
  [\(\sigma\) [T, no edge, name=tsnd]]
  [\(\sigma\ldots\) [\(\text{T}\ldots\), no edge]]]
  \draw (tsnd.north) -- (sfst.south);
\end{forest}\)

\subsection{Diacritical Analysis}
As remarked above, the morphemic analysis still makes an indispensable
assumption about the underlying association even after the removal of
the morphemic association.  This assumption can be removed at the
expense of introducing \enquote{diacritical} features, where the
second tonal term is associated with the first or second syllable
depending on the diacritic.  In other words, the following two
mappings are desired:

\ex.
\a. \(
\begin{forest}
  auto,
  [, phantom
  [\(\sigma\), baseline [T, no edge]]
  [\(\sigma\) [\(\text{T}^{2}\), no edge]]
  [\(\sigma\ldots\) [\(\text{T}\ldots\), no edge]]]
\end{forest}
\to
\begin{forest}
  auto,
  [, phantom
  [\(\sigma\), baseline [T]]
  [\(\sigma\) [\(\text{T}^{2}\)]]
  [\(\sigma\ldots\) [\(\text{T}\ldots\), no edge]]]
\end{forest}\)
\b. \(
\begin{forest}
  auto,
  [, phantom
  [\(\sigma\), baseline [T, no edge]]
  [\(\sigma\) [\(\text{T}^{1}\), no edge]]
  [\(\sigma\ldots\) [\(\text{T}\ldots\), no edge]]]
\end{forest}
\to
\begin{forest}
  auto,
  [, phantom
  [\(\sigma\), baseline, name=sfst [T]]
  [\(\sigma\) [\(\text{T}^{1}\), no edge, name=tsnd]]
  [\(\sigma\ldots\) [\(\text{T}\ldots\), no edge]]]
  \draw (tsnd.north) -- (sfst.south);
\end{forest}\)

The definition of such a transduction is straightforward:

\ex. \(\mathit{Assoc}'(x, y) \coloneq
((\text{T}^{1}(x) \lor \mathit{First}(x))
\land \mathit{First}(y)) \lor
(\text{T}^{2}(x) \land \mathit{Second}(y))\)

assuming that only the second tonal term bears the diacritical
features.  Assuming otherwise requires the addition of
\(\mathit{Second}(x)\), which does not change the overall complexity.

This transduction \textemph{seems} to be simpler.  However, it is
exactly equivalent to the transduction based on underlying
association.  To justify this fact, consider that the two diacritical
features are complementary, that is,

\ex.
\a. \(\text{T}^{1}(x) = \neg\text{T}^{2}(x)\)
\b. \(\text{T}^{2}(x) = \neg\text{T}^{1}(x)\)

Viewed this way, it is very much like subcategorization as used in the
odd--even transformation.  Therefore, an alternative definition for
our transduction is

\ex. \(\mathit{Assoc}'(x, y) \coloneq
((\text{T}^{1}(x) \lor \mathit{First}(x))
\land \mathit{First}(y)) \lor
(\neg\text{T}^{1}(x) \land \mathit{Second}(y))\)

In some sense, underlying association is an alternative encoding for
this diacritical feature:

\ex. \(\text{T}^{1}(x) = (\exists y\ldotp \mathit{Assoc}(x, y)) \land
\mathit{Second}(x)\)

To put it another way, the diacritical feature is \textemph{embedded}
as underlying association.  This reflects the intuition that
exceptional patterns only occur when the second tonal term is not a
floating tone.  The difference is whether such a notion is encoded as
a predicate or binary relation.  Either way, the previous
transformation that feeds tone sandhi must be sensitive to the types
of the first syllable that lead to exceptional patterns.

As we have seen, the many-to-one nature of autosegmental
representations is unavoidable precisely due to the use of binary
relations.  Binary relations are convertible to predicates through
quantification, but doing so essentially creates a surjection, that
is, many-to-one mapping.  Linear representations, on the other hand,
avoid the problem altogether due to the lack of binary relations.  Our
purpose is not to argue for one representation over the other, but
merely to emphasize the essential properties of each representation.

\section{Interim Summary}
A recurring debate in theoretical linguistics is the nature of the
primitives in our theories.  It is often assumed that there is a
fundamental difference between, say, privative and binary features.
From a computational point of view, the primitives \textemph{per se}
are really meaningless, and only their behaviors matter.  In this
section, the two systems in question are linear and autosegmental
representations.

Indeed, what makes autosegmental representations special is their
extra power brought by the association relation, not the fact that
tonal features can be decomposed into multiple terms.  If the
autosegmental transformation can be proved to be equivalent to a
linear transformation, linear representations can as well be used with
no loss in power.  In particular, autosegmental representations
require phonotactic constraints to verify the well-formedness, and
such constraints are very much equivalent to a transformation to
linear representations.

Ironically, more aggressive theoretical proposals have been proposed
that decompose segmental features into multiple terms, such as
\textcite{klv85ispetcg}, but only the autosegmental treatment of tonal
features is widely accepted.  These autosegmental proposals are all
subject to the abovementioned problem in that phonotactic constraints
are required.  In this sense, linear representations simplify the
computation by implicitly encode these constraints, revolving to our
original point.

\chapter{Conclusion}
This chapter summarizes and discusses the findings of this thesis.
The comparison between the computational and theoretical approaches to
phonology is presented, followed by future prospects that aim to
extend on the findings.  See also the interim summaries of the
previous chapters for the theoretical interpretations.

\section{Computation vs.\ Theory}
There is admittedly a gap between computational models and theoretical
proposals.  Computational phonology, at least the school that this
thesis is a part of, focuses on characterizing each phonological
phenomenon in terms of computational devices with as few assumptions
as possible.  Theoretical phonology, on the other hand, strives for a
more elegant and unified explanation of each phonological phenomenon.
This leads to strong, often too strong, assumptions made by
theoretical proposals.

Is this an irreconcilable conflict between the two approaches?  This
thesis is obviously not enough to provide a decisive answer to this
question.  Nonetheless, what we see in the case studies is that the
intuition of theoretical proposals can often be preserved even if we
drop some of the assumptions.  This is one of the aims of this thesis,
that is, to fit the theoretical proposals in the computational world.

To some degree, the computational approach also echoes with the idea
of \enquote{conspiracy} à la \textcite{k70fupr}.  Optimality theory
has been successful precisely because it provides a way to derive
phonological phenomena with few assumptions, namely a constraint
optimizer parameterized by a set of constraints.  The computational
approach assumes equally as few, namely a transducer parameterized by
a set of logical relations or equivalently a set of transitions.

\section{Future Prospect}
\paragraph{Type Theory}
Following \posscite{h69fnc} \enquote{formulae-as-types} conception,
logical formulae correspond to types, and by extension logical calculi
correspond to type systems.  For example, logical conjunction
\(p \land q\) corresponds to product types \(\alpha \times \beta\),
logical disjunction \(p \lor q\) corresponds to sum types
\(\alpha + \beta\), and logical implication \(p \implies q\)
corresponds to function types \(\alpha \to \beta\).  Given the heavy
use of logical formulae in this thesis, one naturally wonders whether
it can benefit from a type-theoretic characterization of transducers.

A type-theoretical approach can potentially lead to a more structured
characterization of phonological transformation by enriching the
types.  At the moment, the input and output types are simply graph
types, specifically single-pathed graphs in the case of linear
sequences.  This means that a large subset of graphs are impossible as
either input or output, resulting in partial functions.  By reflecting
this fact in the type, we \enquote{make illegal states
  unrepresentable} as per the catchy slogan, therefore avoiding
partial functions.

\paragraph{What is Interface?}
Interface is an often referenced but seldom defined concept.  From a
computational perspective, an \textterm{interface} defines what
information can be communicated between two levels.  The two levels
must interact in such a manner that respects the interface, which
forces us to clearly state what each level does.  However, we have
seen that some theoretical proposals entangle morphosyntax and
phonology, contradicting this understanding of interface.

Recently, advances in theoretical morphology have brought to us
several \enquote{decompositional} theories, such as
\posscite{hm93dmpi} distributed morphology and \posscite{c09nc}
nanosyntax.  A computational study of them will hopefully improve our
understanding of the morphosyntax--phonology interface, especially in
contrast to \enquote{process-based} theories such as \textcite{a92am}.

\paragraph{Wishful Thinking}
This thesis adopts a \enquote{wishful thinking} that all defined
logical transductions actually made sense.  Keen-eyed readers will
notice that some of them are neither input nor recursive strictly
local, where both the input and recursive predicates use either the
predecessor or successor function.  This thesis wishes that such
logical transductions made sense by conjecturing that they correspond
to input--output strictly local functions.  Whether this is true and
how this extends to autosegmental representations remain unanswered.

\printbibliography[heading=bibintoc]

\printindex
\end{document}
