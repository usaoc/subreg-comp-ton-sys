% -*- TeX-engine: luatex; -*-
% Copyright (C) 2023 Wing Hei Chan

% This work is licensed under a Creative Commons
% Attribution-ShareAlike 4.0 International license.  To view a copy of
% this license, visit
% <https://creativecommons.org/licenses/by-sa/4.0/>.

\documentclass[12pt, a4paper]{report}
\usepackage{imakeidx}
\usepackage{setspace}
\usepackage[type={CC},modifier={by-sa},version={4.0}]{doclicense}
\usepackage[margin=1in]{geometry}
\usepackage{lua-widow-control}
\usepackage{microtype}
\usepackage{mathtools}
\usepackage{unicode-math}
\usepackage[style]{abstract}
\usepackage[style=unified]{biblatex}
\usepackage[style=british]{csquotes}
\usepackage{datetime}
\usepackage{enumitem}
\usepackage{titlesec}
\usepackage{tikz}
\usepackage[linguistics]{forest}
\usepackage{linguex}

\onehalfspacing
\setlist{noitemsep}
\titleformat{\chapter}{\normalfont\huge\bfseries}{\thechapter}{1em}{}
\titlespacing*{\chapter}{
  0pt}{3.5ex plus 1ex minus .2ex}{2.3ex plus .2ex}

\setmainfont{Libertinus Serif}
\setmathfont{Libertinus Math}
\setmonofont[Scale=MatchUppercase]{Iosevka}

\addbibresource{./reference.bib}

\makeindex[intoc]

\newdateformat{monthyyyy}{\monthname[\THEMONTH] \THEYEAR}

\DeclareNameWrapperFormat{labelname:poss}{#1's}
\newrobustcmd*{\posscitealias}{%
  \AtNextCite{\DeclareNameWrapperAlias{labelname}{labelname:poss}}}
\newrobustcmd*{\posscite}{\posscitealias\textcite}

\usetikzlibrary{automata}
\tikzset{auto}
\forestset{auto/.style={for root={baseline}, for tree={calign=first}}}

\newcommand{\context}{\mathrel{/}}
\newcommand{\cuhk}{The Chinese University of Hong Kong}
\newcommand{\gap}{\underline{\hspace{1em}}}
\newcommand{\textemph}[1]{\textsc{#1}}
\newcommand{\textphon}[1]{[#1]}
\newcommand{\textterm}[1]{\textsc{#1}\index{#1}}

\begin{document}
\pagenumbering{alph}
\begin{titlepage}
  \vspace*{\fill}
  \begin{center}
    \begin{Large}
      \bfseries
      Subregular Complexity of Tonal Systems:

      Case Studies of Chinese Languages
    \end{Large}

    by

    CHAN Wing Hei

    under

    Professor LAI Yee King Regine

    \bigskip

    A Thesis To Be Submitted in Partial Fulfilment of

    the Requirement for the Degree of

    Bachelor of Arts

    in

    Linguistics

    \bigskip

    \cuhk

    \monthyyyy\formatdate{28}{12}{2022}
  \end{center}
  \pretocmd{\doclicenseLongText}{%
    Copyright \copyright\ 2023 Wing Hei Chan\par}{}{}
  \doclicenseThis
  \vspace*{\fill}
\end{titlepage}

\cleardoublepage
\pagenumbering{roman}

\chapter*{Abstract}
\addcontentsline{toc}{chapter}{Abstract}
Phonological transformations based on finite-state transducers have
received an important status since their introduction under the guise
of rewrite rules.  Finite-state transducers represent regular
relations, which are closed under composition.  This shows that an
apparently complex system of phonological rules can be compiled into a
single finite-state transducer.  On the other hand, regular relations
are still too powerful as a probable upper bound of the complexity of
phonological transformations.  For example, regular relations are able
to encode such a phonological transformation that only applies to an
even number of terms, obviously unattested as an adequate rule.

Recently, studies of the so-called subregular classes, that is,
classes even weaker than regular relations, have proved a more
probable upper bound.  Such classes are conceptualized as a hierarchy,
not unlike the traditional Chomsky hierarchy in formal language
theory.  The remaining problem is how many phonological phenomena are
explainable and, more importantly, \textemph{not} explainable by each
class.  In this inquiry, tonal systems post an interesting challenge,
and thus require a more in-depth investigation.

Traditionally, tonal systems are accounted for by nonlinear phonology,
which involves the use of multiple phonological tiers and their
associations.  This process can be carefully separated into two ideas:
%
\begin{enumerate}
\item Tones can be independently represented on a separate tier just
  as segments do;
\item The associations between phonological tiers interact with tonal
  transformations in a nontrivial way.
\end{enumerate}
%
In this thesis, the two ideas are examined under the lens of attested
tonal systems found in Chinese languages with vastly different tonal
transformations.  Computationally, the former is embodied by linear
strictly local functions, and the latter by autosegmental strictly
local functions.  Their implication applies not only to tonal systems,
but also to the interfaces between linguistic levels in general.

\tableofcontents

\cleardoublepage
\pagenumbering{arabic}

\chapter{Theoretical Background}
This chapter introduces the theoretical foundations needed for the
interpretation of tonal systems.  The theories are presented from two
perspectives, namely theoretical and computational phonology, which
utilize different methodologies.  Although this thesis uses a
computational approach, the theoretical implication requires an
understanding of relevant approaches as well.

Specifically, the focus is placed on phonological
\textterm{transformations} as opposed to \textterm{phonotactics}.
Their distinction corresponds to the distinction between transducers
and acceptors in automata theory, which reflect translation and
membership problems respectively.  Loosely speaking, a transducer is a
function of type \(\alpha \to \beta\), while an acceptor is a function
of type \(\alpha \to \mathit{Boolean}\), that is, a
predicate.\footnote{Unless a gradient interpretation of phonotactics
  is required, in which case the output type can be, for example, an
  interval type.}  The latter will be briefly mentioned as the
theories are introduced.

Given the theoretical background, this chapter then explains the
rationales behind the chosen methodology as well as the expected
results.  In particular, this thesis has the goal of clarifying the
unique operations on autosegmental representations that empower them
to account for certain tonal transformations.

\section{Theoretical Phonology}
This section details the presentations of phonological transformations
in theoretical phonology.  Phonological transformations have been
employed by two allegedly divergent formalisms, often named linear and
nonlinear phonology.  As explained in the later section on the
computational perspective, this categorization fails to recognize the
actual properties that render computationally more complex formalisms
necessary.  Nonetheless, this section will follow the traditional
categorization.

This section also discusses the organization of grammar.  As will
become clear once computational models are introduced, organization
only matters in the compositions of phonological transformations,
contrary to the common conception that cyclicity and ordering are
inherent properties of phonological systems.  Lexical phonology is
mentioned for its significance in the arguments against cyclicity.

The discussion of \textterm{optimality theory} as in
\textcite{ps93otcigg} is intentionally omitted.  The reason is not
that optimality theory is a worse formalism, but merely that the
formalization of it results in computational devices out of the scope
of this thesis.  Specifically, optimality theory, due to its
constraint-optimizing nature, requires constraint optimizers rather
than transducers, which are potentially powerful enough to
overgenerate.  For such an implementation based on dynamic
programming, refer to \textcite{t95cot}.  For discussions of the
computational complexity of optimality theory, refer to
\textcite{e97egpot, i06sptotci, hkr09ecot}, among others.

\subsection{Linear Phonology}
The earliest discussion of phonological transformations dates back to
\textcite{ch68spe}, where rules are in the form of context-sensitive
rewrite rules.  For example, a vowel nasalization rule can be in the
form

\ex. \(\text{V} \to [+\text{nasal}]
\context \text{\gap}[+\text{nasal}]\)

read as \enquote{nasalize a vowel immediately before a nasal sound}.
Under the usual notation of rewrite rules, this is written as

\ex. \(\text{V}[+\text{nasal}] \to
(\text{V} \cup [+\text{nasal}])[+\text{nasal}]\)

understood as \enquote{rewrite a vowel immediately before a nasal
  sound to the union of the vowel and the singleton set of the nasal
  feature}.  It should be noted that this apparently context-sensitive
form is by no means actually context-sensitive.  What it means is
that, given the Chomsky hierarchy \parencite{c59cfpg}

\ex. \(\text{Recursively enumerable} \supset
\text{Context-sensitive} \supset
\text{Context-free} \supset
\text{Regular}\)

it is not true that phonological transformations form a superset of
context-free relations.  This is shown in the later section on regular
relations.

Attention should be paid to the term \enquote{linear}.  The
interpretation of the term is that phonological transformations apply
to a linear sequence of segments.  This implies the assumption that
segmental phonology, the sort that is dealt with above, is disjoint
from nonsegmental phonology.  This assumption is computationally
unsound, as nonsegmental phonology does not necessarily require
\enquote{nonlinear} accounts.  Another interpretation has to do with
the notion of locality, which is an often misunderstood notion in
theoretical phonology.

A relevant concern is the representations of tonal features.  Indeed,
due to the \enquote{linear} nature of the representations, tonal
features have nowhere else to fit in but the same feature sets as the
segmental features.  Consider, for example, \posscite{w67pft}
interesting characterization of Xiamen tone cycle:

\ex. \([\mathop{\alpha}\text{high}, \mathop{\beta}\text{fall}]
\to [\mathop{\beta}\text{high}, \mathop{-\alpha}\text{fall}]\)

This rule not only uses the same \textemph{form} as segmental rules,
but also the same \textemph{representation}.  Of course, it is not a
problem if the rule explains the tone cycle well.\footnote{See,
  however, \posscite{c00tspcd} criticism of the characterization.}  On
the other hand, a nonlinear account allows more freedom both in the
form and representation of the rule.  Readers are thus reminded that
representation is a separate matter from the computational complexity
of the formalism \textemph{independent} of representation.

\subsection{Nonlinear Phonology}
Despite the remarks given above, nonsegmental phonology is precisely
what has inspired the development of nonlinear phonology, also known
as autosegmental phonology.  Early applications of nonlinear phonology
can be found in \textcite{c76vhngpam, g76ap}, among others.  Under
nonlinear phonology, phonological representations consist of multiple
associated tiers, each tier with its own linear sequence of terms.
For example, a syllable \textphon{ma} with a high level tone can be
represented as

\ex.
\begin{forest}
  auto, [ma [\(\sigma\) [H] [H]]]
\end{forest}

where \(\sigma\) stands for a syllable and H stands for a high tone.
Consequently, phonological transformations are generalized to operate
on associations.  An unbounded tone spreading rule is:

\ex. \(
\begin{forest}
  auto, [T [\(\sigma\)] [\(\sigma\), no edge] [\(\sigma\), no edge]]
\end{forest}
\to
\begin{forest}
  auto, [T [\(\sigma\)] [\(\sigma\)] [\(\sigma\), no edge]]
\end{forest}
\to
\begin{forest}
  auto, [T [\(\sigma\)] [\(\sigma\)] [\(\sigma\)]]
\end{forest}\)

An implicit assumption of it is that the rule applies iteratively from
left to right.  This intuition is, surprisingly, captured equally well
by linear transformations.  An equally acceptable formulation is the
linear transformation

\ex. \(\text{T}\varepsilon\varepsilon
\to \text{TT}\varepsilon
\to \text{TTT}\)

where \(\varepsilon\) stands for the empty term, assuming the
transformation also applies iteratively from left to right.

The pair of examples above suggests that the power of nonlinear
phonology comes not from the representations, nor from the capability
of iterative application.  Instead, it comes from the ability to
encode the associations in the transformations, as explicified by the
computational models.  Iterative application, on the other hand,
concerns the computational nature of the rules.  As a computer
scientist may say, a transformation is \enquote{mapped} to a sequence
in a certain order, resulting in the apparently long-distance
transformation.

A point should be made that iterative application has little to do
with the organization of grammar.  An iteratively applied phonological
transformation is neither inherently associated with cyclicity nor
with ordering.  In fact, it is not necessarily iterative at all under
an alternative representation.  The only thing relevant is the
\textemph{process} the rule generates as embodied by the equivalent
transducer.

\subsection{Organization of Grammar}
What organization of grammar concerns is, then, \textemph{not} the
computational nature of phonological transformations, but rather the
use of linguistic formalisms to express computationally equivalent
models.\footnote{This is not to deny that linguistic formalisms differ
  in their expressivity in a nontrivial manner.  Expressivity and
  computability are orthogonal, as the famous \enquote{Turing tarpit}
  analogy puts it \parencite{p82ep}.}  This is reminiscent of the use
of programming paradigms to express computationally equivalent
programs given their Turing completeness, except we expect a much
weaker upper bound in phonological models.

The concept of \textterm{cyclicity}, that is, repeatedly applying a
phonological transformation until it is no longer applicable, played
an important role in early formulations of nonsegmental phonology.
The idea is that a series of rules is linearly sequenced as
\(R_{1}, \ldots, R_{n}\), where \(R_{n}\) is a rule that creates new
contexts for the previous rules, often conceived as a \enquote{bracket
  erasure} rule.  This brings two problems:
%
\begin{enumerate}
\item The end result of rule applications is tightly coupled with how
  contexts are represented;
\item It is not clear how to apply different sequences of rules at
  each level of representation.
\end{enumerate}
%
The former is hardly an inherent or unique problem of cyclicity.  The
latter, in comparison, is a more serious problem, as it prohibits an
otherwise modular organization of grammar by the virtue of
\enquote{rules bloat}.  Alternatively, \textterm{lexical phonology}
advocates for a \enquote{stratified} approach \parencite{k82cplp},
thus achieving a more modular organization.  This thesis adopts the
general approach of lexical phonology without assuming any form of the
lexicalist hypothesis.

The \textterm{ordering} of phonological transformations, on the other
hand, is a more universal problem.  Even simply modeling phonological
transformations as functions already leads to the ordering problem, as

\ex. \(\neg(\forall f, g\ldotp f \circ g \implies g \circ f)\)

That is, function composition is noncommutative.  Therefore, this
thesis does not expect this problem to be easily solved.

\section{Computational Phonology}
This section introduces computational formalization of phonological
transformations in terms of regular relations, in particular
subregular functions.  The main method of computationally formalizing
phonological transformations concerns the use of machines in the sense
of automata theory.  In particular, finite-state machines are used for
their limited computational complexity.  They are able to encode
regular relations, but not context-sensitive or even context-free
relations due to their lack of a stack.

Logical encodings of regular relations are also introduced along with
their transductions.  Encoding regular relations in terms of logical
formulae facilitates the understanding of their subparts as compared
to transducers and enables the evaluation of their computational
complexity as first-order, quantifier-free, among others.

\subsection{Regular Relation}
The apparently context-sensitive phonological rules have the important
constraint that all terms are \textterm{terminal}, that is, not able
to be further rewritten into themselves, directly or indirectly.  As
\textcite{c59cfpg} points out, the distinguishing feature of
context-free languages is exactly the ability to self-embed, which
requires recursively rewriting \textterm{nonterminal} terms into
themselves preceded and followed by other terms.  Formally, a grammar
is self-embedding iff

\ex. \(\exists A, \varphi, \psi\ldotp A \Rightarrow \varphi A \psi\)

where neither \(\varphi\) nor \(\psi\) is the identity term and
\(\Rightarrow\) consists of any number of rewrite steps.  Clearly,
phonological rules cannot express such a transformation.
\textcite{kk94rmprs} therefore suggests modeling phonological
transformations as \textterm{regular} relations.

Regular relations are relations whose domains and ranges are regular
languages.  An artificial regular relation, for example, is a relation
that rewrites any sequence that has an \textemph{even} number of
\(a\)s to a sequence with each \(a\) replaced by \(b\).  This is an
unattested phonological transformation, but it is a valid regular
relation, showcasing that regular relations are still computationally
too complex for modeling phonological transformations.

Regular relations can be represented as the equivalent
\textterm{finite-state} transducers.  Finite-state transducers are
machines that transit between a finite number of states according to
the currently read term, where transitions emit the output terms.  To
put it another way, \textterm{acceptors} are machines that accept
\textemph{one} sequence of terms, while \textterm{transducers} are
ones that accept \textemph{two} sequences of terms, namely the input
and output sequences.  The artificial regular relation above requires
a \textterm{nondeterministic} finite-state transducer that either
rewrites \(a\) to \(b\) or does not, whose validity is only determined
after the whole input sequence is read:

\ex.
\begin{tikzpicture}[baseline=(bow.base)]
  \node[state] at (0,0) (bow) {\(q_{1}\)};
  \node[state,accepting] at (2,1) (noop) {\(q_{2a}\)};
  \node[state] at (4,1) (invalid) {\(q_{3a}\)};
  \node[state] at (2,-1) (write) {\(q_{2b}\)};
  \node[state,accepting] at (4,-1) (valid) {\(q_{3b}\)};
  \path[->]
  (bow) edge [loop left] node {\(\text{?}:\text{?}\)} ()
  (bow) edge node {\(a:a\)} (noop)
  (noop) edge [loop above] node {\(\text{?}:\text{?}\)} ()
  (noop) edge [bend left] node {\(a:a\)} (invalid)
  (invalid) edge [loop above] node {\(\text{?}:\text{?}\)} ()
  (invalid) edge [bend left] node {\(a:a\)} (noop)
  (bow) edge ['] node {\(a:b\)} (write)
  (write) edge [loop below] node {\(\text{?}:\text{?}\)} ()
  (write) edge [bend left] node {\(a:b\)} (valid)
  (valid) edge [loop below] node {\(\text{?}:\text{?}\)} ()
  (valid) edge [bend left] node {\(a:b\)} (write);
\end{tikzpicture}

This artificial regular relation is unattested, which means the
transducer overgenerates.  Ideally, two additional constraints are
desired:
%
\begin{enumerate}
\item The transducers be \textterm{deterministic} to exclude multiple
  outputs;
\item The transducers be \textterm{local} to limit the required memory
  resource.
\end{enumerate}
%
These two constraints, in turn, lead to subregular functions.

Alternatively, regular relations can be represented as logical
transductions, specifically \textterm{monadic second-order} ones as
those of \textcite{eh01mdsttft}, which are actually able to represent
nonregular relations.\footnote{An example is the first-order definable
  total reduplication \(w \to ww\), whose output language is included
  in the class of tree-adjoining languages \parencite{sw94efecg}.}
Monadic second-order logic allows quantification over sets, which is
needed for the definition of general precedence \(<\) based on
immediate precedence \(\vartriangleleft\):

\ex.
\a. \(\mathit{Closed}(X) \coloneq \forall x, y\ldotp
(x \in X \land x \vartriangleleft y) \implies y \in X\)
\b. \(x < y \coloneq \forall X\ldotp
(x \in X \land \mathit{Closed}(X)) \implies y \in X\)

This is understood as \enquote{\(x\) precedes \(y\) iff every
  transitive closure of immediate precedence that includes \(x\) also
  includes \(y\)}, which means the minimal one of such transitive
closures also includes \(y\).  A relation is then defined that
expresses that an \(a\) immediately precedes another \(a\) ignoring
other non-\(a\)s between them:

\ex. \(x \vartriangleleft_{a} y \coloneq a(x) \land a(y) \land x < y
\land \neg(\exists z\ldotp a(z) \land x < z \land z < y)\)

The notion of the evenness of the number of \(a\)s is finally defined:

\ex.
\a. \(\mathit{Odd}(x) \coloneq (a(x) \land (\forall y\ldotp
y < x \implies \neg a(y))) \lor (\exists y\ldotp
\mathit{Even}(y) \land y \vartriangleleft_{a} x))\)
\b. \(\mathit{Even}(x) \coloneq \exists y\ldotp
\mathit{Odd}(y) \land y \vartriangleleft_{a} x\)

Accordingly, the transduction is defined:

\ex. \(b'(x) \coloneq b(x) \lor (a(x) \land (\exists y\ldotp
\mathit{Even}(y) \land (\forall z\ldotp y < z \implies \neg a(z))))\)

Similar to how finite-state transducers are constrained, logical
transductions of the sort above are avoided if
%
\begin{itemize}
\item Quantification over sets is disallowed; or
\item Any quantification is disallowed.
\end{itemize}
%
They result in \textterm{first-order} logic and its
\textterm{quantifier-free} version respectively, which correspond to
various subregular functions.

Despite their inadequate computational complexity, regular relations
are useful as they are closed under composition, meaning that

\ex. \(\forall P, Q\ldotp (\mathit{Regular}(P) \land
\mathit{Regular}(Q)) \implies \mathit{Regular}(P \circ Q)\)

This enables the possibility to compile an apparently complex system
of phonological rules into a single finite-state transducer.

\subsection{Subregular Function}
The most trivial class of \textterm{subregular} functions is without a
doubt the class of finite functions, which is disfavored in this
thesis for the reasons mentioned in \textcite{s93wimpatlai}.  Rather,
by enforcing the aforementioned constraints, several nontrivial
classes of subregular functions can be subdivided.  If the
finite-state transducers are required to be deterministic, the
resulting subregular functions are \textterm{subsequential} functions,
further classified into left and right variants depending on the
direction the input sequence is read \parencite{ch12bcsimr, hl13vhs}.
This class is not particularly interesting for the purposes of this
thesis, as it lacks a meaningful notion of locality.

The concept of \textterm{strict locality} is defined for regular
languages, which concerns the recognition of substrings
\parencite{rp11apresh, rhfhlw13csc}.  Borrowing this concept, input
strictly local and output strictly local functions can be defined over
\textterm{linear} representations, which \enquote{remember} a finite
length of input and output substrings respectively
\parencite{ceh14lslsf}.  Similar to subsequential functions, output
strictly local functions can be further classified into left and right
variants \parencite{ceh15oslf}.

An example of input strictly local function is the function that
rewrites \(a\) to \(b\) whenever the last input is \(a\).  That is,

\ex. \(a^{n+1} \to ab^{n}\)

where \(n \in \mathbb{N}\), with the equivalent transducer

\ex.
\begin{tikzpicture}[baseline=(loop.base)]
  \node[state,initial,initial text={\(a:a\)}] at (0,0) (loop) {\(a\)};
  \path[->] (loop) edge [loop right] node {\(a:b\)} ();
\end{tikzpicture}

where the remembered input term is encoded in the state.  More
conveniently, it can also be represented as the logical transduction

\ex.
\a. \(a'(x) \coloneq a(x) \land \neg(a(\mathit{pred}(x)))\)
\b. \(b'(x) \coloneq a(x) \land a(\mathit{pred}(x))\)

where \(\mathit{pred}\) is the predecessor function.

Similarly, an example of left output strictly local function is the
function that rewrites \(a\) to \(b\) whenever the last output is
\(a\).  That is,

\ex.
\a. \(a^{2n} \to (ab)^{n}\)
\b. \(a^{2n+1} \to (ab)^{n}a\)

with the equivalent transducer

\ex.
\begin{tikzpicture}[baseline=(a.base)]
  \node[state,initial,initial text={\(a:a\)}] at (0,0) (a) {\(a\)};
  \node[state] at (2,0) (b) {\(b\)};
  \path[->]
  (a) edge [bend left] node {\(a:b\)} (b)
  (b) edge [bend left] node {\(a:a\)} (a);
\end{tikzpicture}

where the remembered output terms are encoded in the states.  When it
is represented as a logical transduction, however, there occurs
\textterm{recursive} logical formulae,\footnote{Readers familiar with
  fixed-point theorems may expect to see a fixed-point logic
  undermining such recursive logical formulae.  The technical details
  can be found in \textcite{koj18taolns, cj19qlfpfp}.} resulting in a
recursive strictly local function:

\ex.
\a. \(a'(x) \coloneq a(x) \land \neg(a'(\mathit{pred}(x)))\)
\b. \(b'(x) \coloneq a(x) \land a'(\mathit{pred}(x))\)

As \textcite{cj21iolr} conjectures, recursive logical transductions
comprise precisely the logical characterization of output strictly
local functions.

Logical encodings enable the generalization of strict locality over
\textterm{autosegmental} representations \parencite{cj19aislf}.  Under
a predicate logic, associations are understood as a binary relation
\(\mathit{Assoc}\) where \(\mathit{Assoc}(a, b)\) indicates that \(a\)
and \(b\) are associated, given that \(a\) and \(b\) are on separate
tiers.\footnote{A question can be asked whether \(\mathit{Assoc}\) is
  symmetric.  Although this question has no particular bearing,
  \(\mathit{Assoc}\) is assumed to be antisymmetric to allow the
  interpretation of autosegmental representations as directed graphs.}
For example, the unbounded tone spreading rule can be defined as the
recursive logical transduction

\ex. \(\mathit{Assoc}'(x, y) \coloneq \mathit{Assoc}(x, y) \lor
\mathit{Assoc}'(x, \mathit{pred}(y))\)


Strictly local logical tranductions, as shown thus far, satisfy both
first-orderness and quantifier-freeness.  This is not a coincidence,
but rather the expected outcome.  By disallowing quantification,
logical formulae are allowed to refer to terms other than the bounded
ones only by the means of the predecessor and successor functions,
which in turn allude to the notion of locality.

\section{Preliminary Goal}
Chinese languages have been famous for their rich repertoire of tonal
systems and relevant transformations.  In particular, tonal
transformations in Chinese languages often interact with the
morphosyntactic representations.  This fact has inspired analyses that
require morphosyntactic encodings in phonological representations.
This thesis expects to formulate such techniques in terms of strictly
local functions.

Another question this thesis aims to answer is why autosegmental
representations appear necessary.  Indeed, as later shown,
autosegmental representations impose the requirement that phonological
representations be graphs as opposed to linear sequences of terms.
This question is impossible to answer without assessing what
autosegmental representations are useful for.

Finally, this thesis attempts to supply a theoretical interpretation
of the computational analyses.  The theoretical implication of
strictly local functions, this thesis believes, is more general than
the study of computational complexity.  Potentially, it relates to the
more difficult problem of the interfaces between linguistic levels.  A
complete answer to this question requires a better understanding of
the primitives and operations each level allows.

\printbibliography[heading=bibintoc]

\printindex
\end{document}
